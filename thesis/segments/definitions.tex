%%%%%%%%%%%%%%%%%%%%%%%%%%%%%%%%%%%%%%%%%
% Masters/Doctoral Thesis
% LaTeX Template
% Version 3.0 (18/3/22)
%
% This template was downloaded from:
% http://www.LaTeXTemplates.com
%
% Version 2.x major modifications by:
% Vel (vel@latextemplates.com)

% Version 3.x modifications by:
% Felix Friese
%
% This template is based on a template by:
% Steve Gunn (http://users.ecs.soton.ac.uk/srg/softwaretools/document/templates/)
% Sunil Patel (http://www.sunilpatel.co.uk/thesis-template/)
%
% Template license:
% CC BY-NC-SA 3.0 (http://creativecommons.org/licenses/by-nc-sa/3.0/)
%
%%%%%%%%%%%%%%%%%%%%%%%%%%%%%%%%%%%%%%%%%

%----------------------------------------------------------------------------------------
%	PACKAGES AND OTHER DOCUMENT CONFIGURATIONS
%----------------------------------------------------------------------------------------


%\RequirePackage{standalone}
\RequirePackage{sectsty}        % make section heading sans-serif
\RequirePackage{fontawesome5}
\RequirePackage[utf8]{inputenc} % Required for inputting international characters
\RequirePackage[T1]{fontenc} % Output font encoding for international characters
\RequirePackage{mathpazo} % Use the Palatino font by default
\RequirePackage[mathscr]{eucal}
\RequirePackage{microtype}  % somehow make stuff look nicer
\microtypesetup{nopatch=item} % somehow needed
\RequirePackage[backend=bibtex,citestyle=numeric, bibstyle=ieee,natbib=true,sorting=ynt]{biblatex} % Use the bibtex backend
\addbibresource{references.bib} % The filename of the bibliography

\AtEveryBibitem{
    \clearfield{urlyear}
    \clearfield{urlmonth}
    \clearfield{urldate}
}

\RequirePackage{comment}  % comment out sections
% ------------------
% align equations
% ------------------
\RequirePackage{amsmath}  % align equations

%MATHSYMB
\RequirePackage{amssymb}

% ------------------
% wrap figure in text blocks
% ------------------
\RequirePackage{wrapfig}

% ------------------
% fancy plots
% ------------------
\RequirePackage{pgfplots}
\usetikzlibrary{calc, shapes, fit, positioning, arrows.meta, matrix, pgfplots.groupplots}
\usepgfplotslibrary{fillbetween}%, external}
%\tikzexternalize

% use `compat' level 1.3 (or higher) to use the advanced placement features for the axis labels
\pgfplotsset{
    compat=newest
}
% ------------------
% floating stuff, captions and subcaptions, idk
% ------------------
\RequirePackage{float}
\RequirePackage{subcaption}

% ------------------
% very important lorem ipsum package
% ------------------
\RequirePackage{lipsum}

% ------------------
% remark environment
% ------------------
\RequirePackage{amsthm}
\newtheorem*{remark}{Remark}

% ------------------
% code sections
% ------------------
\RequirePackage{fontspec}
\setmonofont{JetBrainsMono}[
Path=./JetbrainsFontFiles/,
Scale=0.85,
Extension = .ttf,
UprightFont=*-Regular,
BoldFont=*-Bold,
ItalicFont=*-Italic,
BoldItalicFont=*-BoldItalic
]
\RequirePackage{minted}



\RequirePackage[autostyle=true]{csquotes} % Required to generate language-dependent quotes in the bibliography



\usemintedstyle{manni}
%----------------------------------------------------------------------------------------
%    COLORS
%----------------------------------------------------------------------------------------
\definecolor{LightGray}{gray}{0.95}
\definecolor{green}{rgb}{0,0.7,0}
\definecolor{modernblue}{HTML}{224fa4}
\definecolor{pyratesdarkred}{HTML}{9f294e}
\definecolor{pyratesgreen}{HTML}{3d734f}
\definecolor{pyratesorange}{HTML}{c76d00}
\definecolor{pyratespurple}{HTML}{473579}


% ------------------
% section number depths
% ------------------
\setcounter{tocdepth}{2}
\setcounter{secnumdepth}{3}

\allsectionsfont{\sffamily}

%----------------------------------------------------------------------------------------
%	MARGIN SETTINGS
%----------------------------------------------------------------------------------------

\geometry{
	paper=a4paper, % Change to letterpaper for US letter
	inner=2.5cm, % Inner margin
	outer=3.8cm, % Outer margin
	bindingoffset=.5cm, % Binding offset
	top=1.5cm, % Top margin
	bottom=1.5cm, % Bottom margin
	%showframe, % Uncomment to show how the type block is set on the page
}

%----------------------------------------------------------------------------------------
%	THESIS INFORMATION
%----------------------------------------------------------------------------------------

% Your thesis title, this is used in the title and abstract, print it elsewhere with \ttitle
\thesistitle{Simulating sedation-induced unconsciousness in a Neural-Mass-Model}
%to improve algorithms for state-of-consciousness detection in patients in the Minimally Conscious State

% Your supervisor's name, this is used in the title page, print it elsewhere with \supname
\supervisor{Dr. Malte \textsc{Schilling}}

% Your examiner's name, this is not currently used anywhere in the template, print it elsewhere with \examname
\examiner{Prof Dr. Helge \textsc{Ritter}}
% Your degree name, this is used in the title page and abstract, print it elsewhere with \degreename
\degree{Master of Science}
% Your name, this is used in the title page and abstract, print it elsewhere with \authorname
\author{Felix \textsc{Friese}}


% Your subject area, this is not currently used anywhere in the template, print it elsewhere with \subjectname
\subject{Intelligent Systems}
% Keywords for your thesis, this is not currently used anywhere in the template, print it elsewhere with \keywordnames
\keywords{EEG, NMM, GABA-A, Propofol, Anaesthesia}
% print it elsewhere with \univname
\university{\href{https://uni-bielefeld.de/}{Bielefeld University}}
% print it elsewhere with \deptname
\department{\href{https://www.uni-bielefeld.de/technische-fakultaet/}{Faculty of Technology}}
% print with \groupname
\group{\href{https://www.uni-bielefeld.de/fakultaeten/technische-fakultaet/forschung/ag-ueberblick/neuroinformatik/}{Neuroinformatics Group}}
%print with \facname
\faculty{\href{https://www.uni-bielefeld.de/technische-fakultaet/}{Faculty of Technology}}

\AtBeginDocument{
\hypersetup{pdftitle=\ttitle} % Set the PDF's title to your title
\hypersetup{pdfauthor=\authorname} % Set the PDF's author to your name
\hypersetup{pdfkeywords=\keywordnames} % Set the PDF's keywords to your keywords
}

\DeclareSIUnit{\molar}{M}

%\usepackage{scrhack}
\RequirePackage{tikz}

\RequirePackage{linegoal}

\RequirePackage{import}
\newcommand\iltodo[1]{
\fcolorbox{orange}{orange!10}{
    \fcolorbox{black}{yellow!70}{\small\faIcon{tasks}~ \texttt{Todo}}
    \hspace{4pt} \texttt{#1}}
}

\newcommand\todo[1]{\par
\noindent\fcolorbox{orange}{orange!10}{\parbox{\linegoal}{
    \fcolorbox{black}{yellow!70}{\small\faIcon{tasks}~ \texttt{Todo}}
    \hspace{4pt} \texttt{#1}}}~\\
}

\newcommand\question[1]{\par
\noindent\fcolorbox{yellow}{yellow!10}{\parbox{\linegoal}{
    \fcolorbox{black}{yellow!70}{\small\faIcon{question}~ \texttt{Question}}
    \hspace{4pt} \texttt{#1}}}~\\
}

\newcommand\incomplete[1]{\par
\noindent\fcolorbox{red}{red!10}{\parbox{\linegoal}{
    \fcolorbox{black}{orange!70}{
        \small\faIcon{exclamation-triangle}~\texttt{Section Incomplete}
    }
\texttt{#1}}}~\\[1em]}


\newcommand{\newremark}[2]{

    \vspace{1em}\par
    \noindent\fcolorbox{yellow}{yellow!10}{
       \begin{minipage}{\linewidth}
        \textbf{Remark} (#1). \textit{ #2 }
       \end{minipage}
    }

    \vspace{1em}\noindent
}

\newcommand{\newremarkimg}[3]{

    \vspace{1em}
    \noindent\fcolorbox{yellow}{yellow!10}{
       \begin{minipage}{\linewidth}
           #2
        \textbf{Remark} (#1). \textit{ #3 }
       \end{minipage}
    }

    \vspace{1em}\noindent
}

\newcommand\citationneeded{\textcolor{blue}{[citation-needed]}}