%%%%%%%%%%%%%%%%%%%%%%%%%%%%%%%%%%%%%%%%%
% Masters/Doctoral Thesis
% LaTeX Template
% Version 2.5 (27/8/17)
%
% This template was downloaded from:
% http://www.LaTeXTemplates.com
%
% Version 2.x major modifications by:
% Vel (vel@latextemplates.com)
%
% This template is based on a template by:
% Steve Gunn (http://users.ecs.soton.ac.uk/srg/softwaretools/document/templates/)
% Sunil Patel (http://www.sunilpatel.co.uk/thesis-template/)
%
% Template license:
% CC BY-NC-SA 3.0 (http://creativecommons.org/licenses/by-nc-sa/3.0/)
%
%%%%%%%%%%%%%%%%%%%%%%%%%%%%%%%%%%%%%%%%%

%----------------------------------------------------------------------------------------
%	PACKAGES AND OTHER DOCUMENT CONFIGURATIONS
%----------------------------------------------------------------------------------------

\documentclass[
11pt, % The default document font size, options: 10pt, 11pt, 12pt
%oneside, % Two side (alternating margins) for binding by default, uncomment to switch to one side
english, % ngerman for German
singlespacing, % Single line spacing, alternatives: onehalfspacing or doublespacing
%draft, % Uncomment to enable draft mode (no pictures, no links, overfull hboxes indicated)
%nolistspacing, % If the document is onehalfspacing or doublespacing, uncomment this to set spacing in lists to single
%liststotoc, % Uncomment to add the list of figures/tables/etc to the table of contents
%toctotoc, % Uncomment to add the main table of contents to the table of contents
%parskip, % Uncomment to add space between paragraphs
%nohyperref, % Uncomment to not load the hyperref package
headsepline, % Uncomment to get a line under the header
%chapterinoneline, % Uncomment to place the chapter title next to the number on one line
%consistentlayout, % Uncomment to change the layout of the declaration, abstract and acknowledgements pages to match the default layout
]{MastersDoctoralThesis} % The class file specifying the document structure

\usepackage{fontawesome5}
\usepackage[utf8]{inputenc} % Required for inputting international characters
\usepackage[T1]{fontenc} % Output font encoding for international characters

\usepackage{mathpazo} % Use the Palatino font by default
\usepackage[mathscr]{eucal}

\usepackage{microtype}  % somehow make stuff look nicer
\microtypesetup{nopatch=item} % somehow needed

\usepackage[margin=15pt,font=small,labelfont={bf,sf}]{caption}  % make better caption fonts
\usepackage{sectsty}        % make section heading sans-serif
\allsectionsfont{\sffamily} % see above

\usepackage[backend=bibtex,style=ieee,natbib=true,sorting=ynt]{biblatex} % Use the bibtex backend with the authoryear-comp citation style (which resembles APA)

\addbibresource{references.bib} % The filename of the bibliography

\usepackage[autostyle=true]{csquotes} % Required to generate language-dependent quotes in the bibliography
% ------------------
% align equations
% ------------------
\usepackage{amsmath}  % align equations

%MATHSYMB
\usepackage{amssymb}

% ------------------
% wrap figure in text blocks
% ------------------
\usepackage{wrapfig}

% ------------------
% fancy plots
% ------------------
\usepackage{pgfplots}
\usetikzlibrary{calc, shapes, positioning, arrows.meta, matrix, pgfplots.groupplots}
\usepgfplotslibrary{fillbetween, external}
\tikzexternalize

% use `compat' level 1.3 (or higher) to use the advanced placement features for the axis labels
\pgfplotsset{
    compat=newest
}
% ------------------
% floating stuff, captions and subcaptions, idk
% ------------------
\usepackage{float}
\usepackage{subcaption}

% ------------------
% very important lorem ipsum package
% ------------------
\usepackage{lipsum}

% ------------------
% remark environment
% ------------------
\usepackage{amsthm}
\newtheorem*{remark}{Remark}

% ------------------
% code sections
% ------------------
\usepackage{fontspec}
\setmonofont{JetBrainsMono}[
Path=./JetbrainsFontFiles/,
Scale=0.85,
Extension = .ttf,
UprightFont=*-Regular,
BoldFont=*-Bold,
ItalicFont=*-Italic,
BoldItalicFont=*-BoldItalic
]


\usepackage{minted}

%\usepackage{xcolor}

\usemintedstyle{manni}


%\usepackage{accsupp}    % for ignoring line-numbers while copy pasting


\definecolor{LightGray}{gray}{0.95}
\definecolor{green}{rgb}{0,0.7,0}
%%
\definecolor{pyratesdarkred}{HTML}{9f294e}
\definecolor{pyratesgreen}{HTML}{3d734f}
\definecolor{pyratesorange}{HTML}{c76d00}
\definecolor{pyratespurple}{HTML}{473579}

%\usepackage{listings}
%\definecolor{codegreen}{rgb}{0,0.6,0}
%\definecolor{codegray}{rgb}{0.5,0.5,0.5}
%\definecolor{codepurple}{rgb}{0.58,0,0.82}
%\definecolor{backcolour}{rgb}{0.95,0.95,0.92}
%\lstdefinestyle{mystyle}{
%	backgroundcolor=\color{backcolour},
%	commentstyle=\color{codegray},
%	keywordstyle=\color{magenta},
%	numberstyle=\tiny\color{codegray}\noncopynumber,
%	identifierstyle=\color{blue},
%	stringstyle=\color{codegreen},
%	basicstyle=\ttfamily\footnotesize,
%	columns = flexible,
%	breakatwhitespace=false,
%	breaklines=true,
%	captionpos=b,
%	keepspaces=true,
%	numbers=left,
%	numbersep=5pt,
%	showspaces=false,
%	showstringspaces=false,
%	showtabs=false,
%	tabsize=4
%}
%\lstset{style=mystyle,columns=fullflexible}
%
%\newcommand{\noncopynumber}[1]{
%	\BeginAccSupp{method=escape,ActualText={}}
%	#1
%	\EndAccSupp{}
%}
%\newcommand\YAMLcolonstyle{\color{magenta}\mdseries}
%\newcommand\YAMLkeystyle{\color{cyan}\bfseries}
%\newcommand\YAMLvaluestyle{\color{black}\mdseries}
%
%\makeatletter
%
%% here is a macro expanding to the name of the language
%% (handy if you decide to change it further down the road)
%\newcommand\language@yaml{yaml}
%
%\expandafter\expandafter\expandafter\lstdefinelanguage
%\expandafter{\language@yaml}
%{
%	%keywords={true,false,null,y,n},
%	keywordstyle=\color{orange}\bfseries,
%	identifierstyle=\color{magenta},
%	basicstyle=\ttfamily\footnotesize\bfseries,      % assuming a key comes first
%	commentstyle=\color{codegray}\mdseries,
%	%sensitive=false,
%	comment=[l]{\#},
%	%morecomment=[s]{/*}{*/},
%	%moredelim=[l][\color{orange}]{\&},
%	%moredelim=[l][\YAMLvaluestyle]{[},
%	%moredelim=**[il][\YAMLkeystyle{\{}\YAMLvaluestyle]{:},   % switch to value style at :
%	%moredelim=**[il][\YAMLcolonstyle{:}\YAMLvaluestyle]{:},   % switch to value style at :
%	morestring=[b]',
%	morestring=[b]",
%	%literate =    {---}{{\ProcessThreeDashes}}3
%	%{>}{{\textcolor{red}\textgreater}}1
%	%{|}{{\textcolor{red}\textbar}}1
%	%{\ -\ }{{\mdseries\ -\ }}3,
%}
%
%% switch to key style at EOL
%%\lst@AddToHook{EveryLine}{\ifx\lst@language\language@yaml\YAMLkeystyle\fi}
%%\makeatother
%
%%\newcommand\ProcessThreeDashes{\llap{\color{cyan}\mdseries-{-}-}}


% ------------------
% section number depths
% ------------------
\setcounter{tocdepth}{3}
\setcounter{secnumdepth}{3}

%----------------------------------------------------------------------------------------
%	MARGIN SETTINGS
%----------------------------------------------------------------------------------------

\geometry{
	paper=a4paper, % Change to letterpaper for US letter
	inner=2.5cm, % Inner margin
	outer=3.8cm, % Outer margin
	bindingoffset=.5cm, % Binding offset
	top=1.5cm, % Top margin
	bottom=1.5cm, % Bottom margin
	%showframe, % Uncomment to show how the type block is set on the page
}

%----------------------------------------------------------------------------------------
%	THESIS INFORMATION
%----------------------------------------------------------------------------------------

\thesistitle{Simulating sedation-induced unconsciousness in a Neural-Mass-Model to improve algorithms for state-of-consciousness detection in patients with unresponsive wakefulness syndrome} % Your thesis title, this is used in the title and abstract, print it elsewhere with \ttitle


%%% OLD
%\supervisor{Dipl. Inform. Andrea \textsc{Finke}} % Your supervisor's name, this is used in the title page, print it elsewhere with \supname
%\examiner{Dr. Malte \textsc{Schilling}} % Your examiner's name, this is not currently used anywhere in the template, print it elsewhere with \examname


\supervisor{Dr. Malte \textsc{Schilling}} % Your supervisor's name, this is used in the title page, print it elsewhere with \supname
\examiner{Prof Dr. Helge \textsc{Ritter}} % Your examiner's name, this is not currently used anywhere in the template, print it elsewhere with \examname




\degree{Master of Science} % Your degree name, this is used in the title page and abstract, print it elsewhere with \degreename
\author{Felix \textsc{Friese}} % Your name, this is used in the title page and abstract, print it elsewhere with \authorname
\addresses{} % Your address, this is not currently used anywhere in the template, print it elsewhere with \addressname

\subject{Intelligent Systems} % Your subject area, this is not currently used anywhere in the template, print it elsewhere with \subjectname
\keywords{EEG, NMM, GABA, Anaesthesia, DOC} % Keywords for your thesis, this is not currently used anywhere in the template, print it elsewhere with \keywordnames
\university{\href{https://uni-bielefeld.de/}{Bielefeld University}} % Your university's name and URL, this is used in the title page and abstract, print it elsewhere with \univname
\department{\href{https://www.uni-bielefeld.de/technische-fakultaet/}{Faculty of Technology}} % Your department's name and URL, this is used in the title page and abstract, print it elsewhere with \deptname
\group{\href{https://www.uni-bielefeld.de/(en)/technische-fakultaet/arbeitsgruppen/Neuroinformatik/}{Neuroinformatics Groups}} % Your research group's name and URL, this is used in the title page, print it elsewhere with \groupname
\faculty{\href{https://www.uni-bielefeld.de/technische-fakultaet/}{Faculty of Technology}} % Your faculty's name and URL, this is used in the title page and abstract, print it elsewhere with \facname

\AtBeginDocument{
\hypersetup{pdftitle=\ttitle} % Set the PDF's title to your title
\hypersetup{pdfauthor=\authorname} % Set the PDF's author to your name
\hypersetup{pdfkeywords=\keywordnames} % Set the PDF's keywords to your keywords
}

%\usepackage{scrhack}
\usepackage{tikz}



\newcommand\todo[1]{\fcolorbox{orange}{orange!10}{\parbox{\textwidth}{\textit{\textbf{TODO:} #1}}}}
\newcommand\incomplete[1]{\fcolorbox{orange}{orange!10}{\parbox{\textwidth}{\textbf{Section Incomplete} #1}}\\[1em]}
