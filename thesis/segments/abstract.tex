\begin{abstract}
\addchaptertocentry{\abstractname} % Add the abstract to the table of contents
    Understanding the mechanisms behind emergence, modulation and loss of consciousness in the brain
    is still within the bounds of fundamental research.
    Simulation of predictable processes like general anaesthesia in computer models can help to uncover
    which mechanisms play a role in inducing these state-changes.
    Neural-field models have been shown to be able to reproduce the phenomena of drug-hysteresis and
    temporal EEG power-surges during induction and emergence of anaesthesia.
    To find the simplest possible model capable of producing these effects,
    the basic Jansen-Rit neural-mass model was implemented alongside its subpopulation-extension by David \& Friston.
    Propofol anaesthesia was simulated in both model variants by modulating IPSP decay-time.
    The David-Friston extension was able to reproduce both key phenomena,
    while the basic Jansen-Rit model reproduced the biphasic effect.
    The results substantiate the theory that these effects might be at least partially caused by inherent brain-dynamics
    on population level,
    and further suggest that these dynamics might modulate the preconditions for the emergence of consciousness.
    Additionally, the existence of hysteresis in computer models reinforces the concept of neural inertia.
\end{abstract}