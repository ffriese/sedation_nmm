\begin{abstract}
\addchaptertocentry{\abstractname} % Add the abstract to the table of contents
\todo{This is old! Re-write the abstract to match the new hypothesis! (very last thing to do)}
\textcolor{gray}{\textit{Patients with severe Traumatic Brain Injuries (TBI) often remain in a state of unresponsive
wakefulness.
Brain-Computer-Interface (BCI)-based systems promise to improve the state assessment and to open a
communication channel for patients to express their intent while in conscious states.
Developing such a BCI-System (e.g., with EEG), including the necessary algorithms to assess a patients current
wakefulness or consciousness state from EEG data is a challenging task.
Development, testing and evaluation of these algorithms requires labeled data (ground truth),
which is almost impossible to obtain given the patients' lack of communication capabilities.
Therefore, it would be desirable to generate a synthetic signal, which should ideally resemble
real EEG data in all relevant features.
We previously developed a simple ICA-based model, which generates a multichannel EEG from base-signals with
configurable spectral features.
While this proved useful for testing numerous components of our signal-analysis framework,
it lacks biological plausibility and explanatory power to model the changes in the signal's
properties given an altered state of consciousness.
In this thesis, we propose an approach towards overcoming these issues while sticking with the original goal of
generating realistic, practically useful surrogate data.
A biologically motivated Neural Mass Model (NMM) on cortical-column level is implemented,
which is able to approximate the effects of sedation-induced unconsciousness on the generated signal.
The model is then shown to be able to reproduce the characteristic effects
that sedation has on the EEG-Signals of real subjects.
This is a first step to a model of consciousness-altering processes in the brain,
which could ultimately be extended to realistically simulate other processes like sleep and trauma-induced DoC,
facilitating better detection algorithms and furthering the goal to develop working BCI-Systems in the given context.}}
\end{abstract}