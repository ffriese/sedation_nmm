\section{General Anaesthesia}\label{sec:general-anaesthesia}
Having established the theoretical background for our models,
we can approach \ref{goal:implement_propofol} by investigating the effects of propofol in the context of anaesthesia.
GA is usually employed in surgical contexts to keep patients from experiencing pain during medical
procedures, e.g., an operation.
Sedative drugs are carefully administered to the patient by trained practitioners to induce loss of consciousness,
without causing permanent damage to the brain or other parts of the body.
Propofol is the most popular sedative for GA ~\cite{miner_clinical_2007, sahinovic_clinical_2018}
and its effects on the brain have been extensively studied,
which makes it a good candidate for our simulation goals.

\subsubsection{Realistic propofol concentrations during GA}\label{subsubsec:realistic-prop-conc-during-ga}
To estimate sensible parameter ranges for our model,
it is helpful to establish realistic dosages of propofol in clinical practice.
Some works (e.g,~\cite{iwakiri_individual_2005, ferreira_patterns_2020}) state propofol concentrations in $\SI{}{\frac{\micro\gram}{\milli\litre}}$,
while others (e.g.,~\cite{kitamura_effects_2003, mcdougall_propofol_2008}) report values in $\SI{}{\micro\molar}$ (micromolar =
$\SI{}{\frac{\micro\mol}{\litre}}$).
To get comparable numbers, we first need to establish the following relation for propofol
(molar mass: $\SI{178.27}{\gram}$):
\[ \SI{1}{\frac{\micro\gram\hspace{0.5em}\text{\tiny{Propofol}}}{\milli\litre}}  =
\frac{ \SI{1}{\frac{\micro\gram}{\milli\litre}}}{\SI{178.27}{\frac{\gram}{\mol}}} \approxeq \SI{5.609}{\micro\molar} \]
In this work, we will use $\SI{}{\micro\molar}$ as our default unit
but additionally provide original values in $\SI{}{\frac{\micro\gram}{\milli\litre}}$ where they are taken from a source.

Effect-site concentrations ($c_{e}$, concentration near the synaptic receptors) of propofol may easily range
up to $\SI{5}{\frac{\micro\gram}{\milli\litre}} (\approx\SI{28}{\micro\molar})$.
Loss of consciousness (LOC) occurs on average at $c_e \sim\SI{2.0}{\frac{\micro\gram}{\milli\litre}} (\approx\SI{11
.2}{\micro\molar})$,
while the recovery of consciousness (ROC) averages at $c_e \sim\SI{1.8}{\frac{\micro\gram}{\milli\litre}}
(\approx\SI{10.1}{\micro\molar})$.
Both values may vary substantially for individual subjects.
LOC has a strong tendency to occur at higher concentrations than ROC.~\cite{iwakiri_individual_2005,
    ferreira_patterns_2020}
Throughout GA, effect-site concentration is commonly derived from measured
blood-plasma concentration ($c_p$) using more or less complex Pharmacokinetic (PK)-Models
(e.g.,~\cite{eleveld_general_2014, liang_pharmacokinetics-neural_2015}),
as direct measurement of $c_{e}$ in the brain is impractical for obvious reasons.
As our model requires the concentration at the receptors, it will use $c_{e}$ directly, however,
the role of PK-Models with respect to reported effect-site concentrations in many works is crucial
and important to acknowledge.


\subsubsection{Effects of propofol on the inhibitory PSP}
Propofol is a GABA\textsubscript{A} receptor agonist, i.e, it potentiates the effect of inhibitory
GABA neurotransmitters at GABA\textsubscript{A} receptors and thereby modulates
the IPSP~\cite{sahinovic_clinical_2018}.
Research on the effect of propofol on the IPSC (Inhibitory Post-Synaptic Current) and EPSC
of cortical neurons has shown that propofol strongly affects the IPSP decay time.
The EPSP and the amplitude of the IPSP of these neurons are unaffected by propofol.
Effect-site concentrations at clinically relevant levels increase the IPSP decay time
significantly (e.g., around $\SI{10}{\micro\molar}$ the decay time roughly doubles).
\cite{kitamura_effects_2003, mcdougall_propofol_2008}


\subsubsection{Biphasic Effect}\label{subsubsec:biphasic-effect}
A biphasic effect (an initial increase of an effect, that decreases with higher concentrations) in the EEG can be
observed for many sedatives.
At low drug-concentrations during induction (and roughly during loss of consciousness)
there are surges of brain-activity,
which disappear with further increase of the dosage and the onset of the comatose state.
Similar observations have been made during the recovery of consciousness,
when the steadily declining levels of the anaesthetic agent temporarily cause pronounced brain-activity before the
subject fully regains consciousness.
For propofol, a temporary steep increase in EEG amplitude,
loosely correlated with the onset of LOC, as well as ROC can be observed.
\cite{kuizenga_quantitative_1998, kuizenga_biphasic_2001}
% Stage-transition, unstable area between two stable states (consciousness/unconsciousness)


\subsubsection{Hysteresis of propofol}\label{subsubsec:hysteresis}

If the state of a system depends not only on its parameters, but also the systems' history,
this dependency is called hysteresis.
The human body often reacts differently to the same concentration of a drug,
depending on whether the concentration is rising or decaying.
Hysteresis is well documented during propofol-induced GA~\cite{kuizenga_quantitative_1998,
    iwakiri_individual_2005,sepulveda_evidence_2018,ferreira_patterns_2020, su_hysteresis_2020}.
The most prominent effect is a counter-clockwise hysteresis for LOC and ROC (as mentioned
in~\ref{subsubsec:realistic-prop-conc-during-ga});
The loss on responsiveness of subjects usually start at higher concentrations than its return.
While some of that effect might be caused by inaccurate PK-Models,
misgauging the actual effect-site concentration,
there is a growing body of research supporting the notion of neural inertia
(the brain's resistance to state changes)~\cite{su_hysteresis_2020, ferreira_patterns_2020, luppi_inert_2021}.
Nonetheless, doubts remain with respect to the origin
of the observed time-lag~\cite{mckay_pharmacokinetic_pharmacodynamic_2006, sepulveda_evidence_2018}.
% Therories of reasons: - re-initiation more complex than shutdown, phase-transitions
