\section{EEG}\label{sec:eeg}
The electroencephalogram (EEG) is a technique to measure brain activity.
It is obtained by placing electrodes on the scalp and measuring the voltage differences to a reference electrode.

\subsubsection{Measurement}\label{subsubsec:measurement}
There are multiple factors that hinder measuring brain-activity on the scalp;
First, the activity of single neurons is far too weak to be measurable through the skull and brain-tissue.
This means that there have to be many neurons producing more or less the same signal,
enabling their voltage to add up to measurable amounts, instead of canceling out.
A similar problem is posed by the spatial orientation of individual neurons.
Many types of brain-cells are randomly oriented in the brain,
which leads to their signals canceling out even if they produce synchronized activity.
The fact that we can measure meaningful EEG activity is largely owed to a specific type of cell,
called pyramidal cell,
which happens to form populations that are mostly oriented perpendicular to the surface of the brain.
A recorded EEG signal consists mainly of the summed postsynaptic potentials (PSPs) of pyramidal cell populations
close to the surface of the cortex.
\cite[9--20]{mecarelli_clinical_2019}

\subsubsection{Advantages/Disadvantages}\label{subsubsec:advantages/disadvantages}
The non-invasive nature of the EEG (compared to surgically inserting electrodes directly in the brain),
as well as its cost-effectiveness and ease of use,
make it one of the most useful and accessible neuroimaging tools.
Additionally, the sub-millisecond temporal resolution enables observations that are impossible with techniques like
the fMRI.
However, it also comes with severe limitations;
The indirect measurement through the skull leads to
\qquad \todo{spatial/temporal resolution, invasiveness, noise, ...}
\todo{removal of artifacts}

\subsubsection{Analysis of EEG Signals}
\begin{wraptable}{r}{5.5cm}
    \centering
    \begin{tabular}{ |c|c| }
        \hline
        \textbf{Band name} & \textbf{Range} \\
        \hline
        \hline
        Alpha ($\alpha$) & $8-\SI{13}{\hertz}$ \\
        \hline
        Beta ($\beta$) & $13-\SI{30}{\hertz}$ \\
        \hline
        Gamma ($\gamma$) & $>\SI{30}{\hertz}$ \\
        \hline
        Theta ($\theta$) & $4-\SI{8}{\hertz}$ \\
        \hline
        Delta ($\delta$) & $0.5-\SI{4}{\hertz}$ \\

        \hline
    \end{tabular}
    \caption{\textbf{EEG Frequency Bands}}
    \label{table:freq-bands}
\end{wraptable}
EEG signals are typically analyzed with regard to their composition of frequencies over time.
The resulting spectrograms are often discomposed into specific ranges, called frequency bands,
which are associated with different functional relevance.
The commonly used ranges and their names are summarized in Table~\ref{table:freq-bands}.
One of the most common interpretations of the frequency-bands is their relation to states of consciousness.
For instance, the $\alpha$-band is linked to an awake resting state,
while the $\delta$-band is linked to deep sleep or coma.
Usually, these measures become more meaningful when the frequency band powers are analyzed in relation to each other,
rather than with respect to absolute values.
\todo{explain spectrogram}

\subsubsection{States of Consciousness in the EEG Signal}\label{subsubsec:states-of-consciousness-in-the-eeg-signal}
\qquad \todo{how do the states differ in the signal}








\subsubsection{Simulation}\label{subsec:simulation}

 \qquad \qquad \todo{what can we hope to achieve by simulating an EEG signal}
 \qquad \qquad \todo{which tools are at our disposal (naive frequency mixing,
    population models, simulating individual neurons, ...)}
 \qquad \qquad \todo{argue about models => why did we land on NMMs/population models?}
 
% \begin{itemize}
%	\item What does the EEG measure?
%	\item Which advantages and disadvantages come with this measuring method?
%	\item How do different states of consciousness, especially sedation, appear in the EEG signal?
%	\item How can these states be detected automatically?
%	\begin{itemize}
%		\item How well does that work, where does it fail?
%		\begin{itemize}
%			\item Why is automatic detection hard in subjects with DOC?
%		\end{itemize}
%	\end{itemize}
% \end{itemize}