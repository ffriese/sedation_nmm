\section{EEG}\label{sec:eeg}
\begin{wraptable}{r}{5.5cm}
    \centering
    \begin{tabular}{ |c|c| }
        \hline
        \textbf{Band name} & \textbf{Range} \\
        \hline
        \hline
        Alpha ($\alpha$) & $8-\SI{13}{\hertz}$ \\
        \hline
        Beta ($\beta$) & $13-\SI{30}{\hertz}$ \\
        \hline
        Gamma ($\gamma$) & $>\SI{30}{\hertz}$ \\
        \hline
        Theta ($\theta$) & $4-\SI{8}{\hertz}$ \\
        \hline
        Delta ($\delta$) & $0.5-\SI{4}{\hertz}$ \\

        \hline
    \end{tabular}
    \caption{\textbf{EEG Frequency Bands}}
    \label{table:freq-bands}
\end{wraptable}
The electroencephalogram (EEG) is obtained by placing electrodes on the scalp and
measuring the voltage differences to a reference electrode.
%\subsubsection{Measurement}\label{subsubsec:measurement}
When recording or analyzing EEG signals,
it is important to understand their origin.
There are multiple factors that complicate measuring brain-activity on the scalp:
First, the activity of single neurons is far too weak to be measurable through the skull and brain-tissue.
This means that there have to be many neurons producing more or less the same signal,
enabling their voltage to add up to measurable amounts, instead of canceling out.
A similar problem is posed by the spatial orientation of individual neurons;
Many types of brain-cells are randomly oriented in the brain,
which leads to their signals canceling out even if they produce synchronized activity.
The fact that we can measure meaningful EEG activity is largely owed to a specific type of cell,
called pyramidal cell,
which happens to form populations that are mostly oriented perpendicular to the surface of the brain.
A recorded EEG signal therefore consists mainly of the summed postsynaptic potentials (PSPs) of pyramidal cell populations
close to the surface of the cortex.
\cite[9--20]{mecarelli_clinical_2019}


\subsubsection{Analysis of EEG Signals}
EEG signals are typically analysed with regard to their composition of frequencies over time.
The resulting spectrograms are often grouped into specific ranges, called frequency bands,
which are associated with different functional relevance.
The commonly used ranges and their names \citationneeded are summarized in Table~\ref{table:freq-bands}.
One of the most common interpretations of the frequency-bands is their relation to states of consciousness.
For instance, the $\alpha$-band is linked to an awake resting state,
while the $\delta$-band is linked to deep sleep or coma. \citationneeded
In many cases, these measures become more meaningful when the frequency band powers are analysed in relation to each
other,
rather than with respect to absolute values. \citationneeded

%\todo{explain spectrogram}

