

%----------------------------------------------------------------------------------------
%	UNDERSTANDING THE PSP BLOCK DIFFERENTIAL EQUATIONS
%----------------------------------------------------------------------------------------
\pagebreak
.
\pagebreak
\section{comparision of psp blocks}


\subsection{\parencite{wilson_excitatory_1972}}

\begin{table}[H]
	\centering
	\begin{tabular}{ |c|c|c|c| } 
		\hline
		\multicolumn{2}{|c|}{Parameter}  & Sensible Value & Unit \\
		\hline
		\hline
		prop. of exc. cells firing at $t$ & \(E(t)\) & \( \) & \( \) \\
		\hline
		prop. of inh. cells firing at $t$ & \(I(t)\) & \( \) & \( \) \\
		\hline
		absolute refractory period & $r$ & 4 & $ms$ \\
		\hline
		firing threshold & $\theta$ & & $mV$ \\
		\hline
		propagation delay & $\tau$ & 10 & $ms$ \\
		\hline
		stimulation decay & $\alpha$ & & $mV/s$ (?) \\
		\hline
	\end{tabular}
	\caption{Parameters of \parencite{wilson_excitatory_1972}}
	\label{table:params_wilsoncowan}
\end{table}

Proportion of cells that are excitable at $t$:
\begin{equation}
	1 - \underbrace{\int_{t-r}^{t} E(t') dt'}_{\text{refractory cells}}
\end{equation}

Subpopulation response function (Sigmoid):
\begin{equation}
	\mathscr{S}(x) = \int_{0}^{x(t)} D(\theta) d\theta 
\end{equation}

Excitation in exc. cell
\begin{equation}
	E(t+\tau) = \overbrace{(1 - \underbrace{\int_{t-r}^{t} E(t') dt'}_{\text{refractory cells}}}^{\text{excitable cells}}) \cdot \mathscr{S}_e \cdot \int_{-\infty}^{t}\alpha(t-t') \cdot [C_1E(t') - C_2(I(t') + P(t')] dt'
\end{equation}


\subsection{\parencite{zetterberg_performance_1978} (referencing a model developed in 1973)}

\subsection{\parencite{lopes_da_silva_model_1974}}



\subsection{\parencite{jansen_electroencephalogram_1995}}


\begin{table}[H]
	\centering
	\begin{tabular}{ |c|c|c|c| } 
		\hline
		\multicolumn{2}{|c|}{Parameter}  & Default Value & Unit \\
		\hline
		\hline
		Excitatory synaptic gain & \(A\) & \(3.25\) & \(mV\) \\
		\hline
		Lumped repr. of sum of exc. delays & \(a\) & \(100\) & \(Hz\) \\
		\hline
		Inhibitory synaptic gain & \(B\) & \(22\) & \(mV\) \\
		\hline
		Lumped repr. of sum of inh. delays & \(b\) & \(50\) & \(Hz\) \\
		\hline
	\end{tabular}
	\caption{Parameters of the PSP Blocks2}
	\label{table:psp_params2}
\end{table}

The differential equations for the Second Order System  of this model can be derived by transformations in the Laplace-Domain (see \ref{eq:laplace_domain2}). To obtain the required Transfer-Function, we need the Laplace transform $H_e(s)$ of our response function $h_e(t)$. This is shown in \ref{eq:laplace_h_e2}.


\begin{equation}
	H_e(s) =\mathscr{L}\{h_e(t)\}  = \mathscr{L}\{Aate^{-at} \} = \frac{Aa}{(s+a)^2} = \frac{Aa}{s^2+2as+a^2}\label{eq:laplace_h_e2}
\end{equation}



\begin{alignat}{5}
	&                                           & &&          \overbrace{Y(s)}^{\text{output}} \quad &&=& \quad \overbrace{H_e(s)}^{\text{transfer function}} \overbrace{X(s)}^{\text{input}} \label{eq:laplace_domain2} \\
	&  \iff                                     & &&                             Y(s) \quad &&=& \quad \frac{Aa}{s^2+2as+a^2} X(s) \nonumber \\ 
	&  \iff                                     & &&               (s^2+2as+a^2) Y(s) \quad &&=& \quad AaX(s) \nonumber \\ 
	&  \iff                                     & &&          s^2Y(s)+2asY(s)+a^2Y(s) \quad &&=& \quad AaX(s) \nonumber \\ 
	&  \stackrel{\mathscr{L}^{-1}}{\iff} \qquad & && \ddot{y}(t)+2a\dot{y}(t)+a^2y(t) \quad &&=& \quad Aax(t) \nonumber \\ 
	&  \iff                                     & &&                      \ddot{y}(t) \quad &&=& \quad Aax(t)-2a\dot{y}(t)-a^2y(t)  \label{eq:sec_ord_nmm2} \\[1em]
	&                                           & \omit\rlap{which can be expressed as two first order equations:}                 \nonumber \\[1em]
	&                                           & &&                       \dot{y}(t) \quad &&=& \quad z(t) \nonumber \\ 
	&                                           & &&                       \dot{z}(t) \quad &&=& \quad Aax(t)-2az(t)-a^2y(t) \nonumber 
\end{alignat}

Comparing (\ref{eq:sec_ord_nmm2}) to the General Form of Second Order Systems (\ref{eq:gen_sec_order_sys2}), we can see that this is a critically damped system ($\zeta = 1$) and the natural frequency $\tau$ of the system is set at $a$.

\begin{align}
	f(t) &= \ddot{y}(t)+ 2\zeta \tau\dot{y}(t) + \tau^2y(t) \nonumber \\ 
	\iff  \ddot{y}(t) &= f(t) - 2\zeta \tau\dot{y}(t) - \tau^2y(t)  \label{eq:gen_sec_order_sys2}
\end{align}

\begin{figure}[H]
	\centering
	\includegraphics[width=12cm]{Figures/jansenrit/jansen_rit_ode_graph.png}
	\caption{Simplified Model after \parencite{jansen_electroencephalogram_1995}}
	\label{Fig: Jansen Rit Simple2}
\end{figure}

Taking the two equations for $\dot{y}$ and $\dot{z}$ as a base, we can now state the equations for the full NMM with it's three populations:

\begin{equation}
	\begin{aligned}
		\dot{y}_0(t) &= y_3(t) \\
		\dot{y}_3(t) &= \fcolorbox{green}{white!30}{$ Aa Sigm[y_1(t) - y_2(t)] - 2ay_3(t) - a^2y_0(t) $}\\
		\dot{y}_1(t) &= y_4(t) \\
		\dot{y}_4(t) &= \fcolorbox{red}{white!30}{$ Aa(p(t) + C_2Sigm[C_1y_0(t)]) - 2ay_4(t) - a^2y_1(t)$}\\
		\dot{y}_2(t) &= y_5(t) \\
		\dot{y}_5(t) &= \fcolorbox{blue}{white!30}{$ Bb(C_4Sigm[C_3y_0(t)]) - 2by_5(t) -b^2y_2(t)$} \\
	\end{aligned}
\end{equation}

\begin{equation}
	\begin{aligned}
		\frac{d}{dt}PSP_{EIN} &= PSP_{t_{EIN}} \\
		\frac{d}{dt}PSP_{t_{EIN}} &= \fcolorbox{lime}{white!30}{$ \frac{h_e}{\tau_e}\cdot C_1 Sigm[PSP_{PC}]  -\frac{2}{\tau_e} \cdot PSP_{t_{EIN}} - \left(\frac{1}{\tau_e}\right)^2 \cdot PSP_{EIN} $}\\
		\frac{d}{dt}PSP_{IIN} &= PSP_{t_{IIN}} \\
		\frac{d}{dt}PSP_{t_{IIN}} &= \fcolorbox{teal}{white!30}{$ \frac{h_e}{\tau_e} \cdot C_3 Sigm[PSP_{PC}]  -\frac{2}{\tau_e} \cdot PSP_{t_{IIN}} - \left(\frac{1}{\tau_e}\right)^2 \cdot PSP_{IIN} $}\\
		\frac{d}{dt}PSP_{PCE} &= PSP_{t_{PCE}} \\
		\frac{d}{dt}PSP_{t_{PCE}} &= \fcolorbox{red}{white!30}{$ \frac{h_e}{\tau_e} \cdot (u + C_2 Sigm[PSP_{EIN}])  -\frac{2}{\tau_e} \cdot PSP_{t_{PCE}} - \left(\frac{1}{\tau_e}\right)^2 \cdot PSP_{PCE} $}\\
		\frac{d}{dt}PSP_{PCI} &= PSP_{t_{PCI}} \\
		\frac{d}{dt}PSP_{t_{PCI}} &= \fcolorbox{blue}{white!30}{$ \frac{h_i}{\tau_i} \cdot C_4 Sigm[PSP_{IIN}])  -\frac{2}{\tau_i} \cdot PSP_{t_{PCI}} - \left(\frac{1}{\tau_i}\right)^2 \cdot PSP_{PCI} $}\\
		PSP_{PC} &= PSP_{PCE} - PSP_{PCI}\\
	\end{aligned}
\end{equation}    

\begin{figure}[H]
	\centering
	\includegraphics[width=12cm]{Figures/jansenrit/pyrates_ode_graph.png}
	\caption{Jansen-Rit Model with Modules in PyRates \parencite{gast_pyratespython_2019}}
	\label{Fig: PyratesODE}
\end{figure}

\subsection{\parencite{gast_pyratespython_2019}}
\begin{table}[H]
	\centering
	\begin{tabular}{ |c|c|c|c|c| } 
		\hline
		\multicolumn{2}{|c|}{Parameter}  & Value & Unit & Relation to \parencite{jansen_electroencephalogram_1995} \\
		\hline
		\hline
		\rule{0pt}{3ex}Excitatory delays & \(\tau_e\) & \(0.01\) & $s$ & $ a = \frac{1}{\tau_e} $ \\[1.2ex]
		\hline
		\rule{0pt}{3ex}Inhibitory delays & \(\tau_i\) & \(0.02\) & $s$ & $ b = \frac{1}{\tau_i} $\\[1.2ex]
		\hline
		\rule{0pt}{3ex}Excitatory synaptic gain & \(h_e\) & \(3.25\) & $mV$ & $ A = h_e $ \\[1.2ex]
		\hline
		\rule{0pt}{3ex}Inhibitory synaptic gain & \(h_i\) & \(22\) & $mV$ & $ B = h_i $ \\[1.2ex]
		\hline
		\rule{0pt}{3ex}PSP of $pop$ & \(PSP_{pop}\) &  & & $ PSP_{pop} = y  $ \\[1.2ex]
		\hline
		\rule{0pt}{3ex} Deriv. of PSP of $pop$  & \(PSP_{t_{pop}}\) &  & & $\frac{d}{dt}PSP_{pop} = PSP_{t_{pop}} = z = \dot{y}$ \\[1.2ex]
		\hline
	\end{tabular}
	\caption{Parameters of the PyRates ODEs}
\end{table}


$y_0 = PSP_{PC}, y_1 ~= PSP_{PCE}, y_2 ~= PSP_{PCI}, y_3=?$ 

\begin{equation}
	\begin{aligned}
		\frac{d}{dt}PSP_{EIN} &= PSP_{t_{EIN}} \\
		\frac{d}{dt}PSP_{t_{EIN}} &= \frac{h_e}{\tau_e}\cdot C_1 Sigm[PSP_{PC}]  -\frac{2}{\tau_e} \cdot PSP_{t_{EIN}} - \left(\frac{1}{\tau_e}\right)^2 \cdot PSP_{EIN} \\
		\frac{d}{dt}PSP_{IIN} &= PSP_{t_{IIN}} \\
		\frac{d}{dt}PSP_{t_{IIN}} &= \frac{h_e}{\tau_e} \cdot C_3 Sigm[PSP_{PC}]  -\frac{2}{\tau_e} \cdot PSP_{t_{IIN}} - \left(\frac{1}{\tau_e}\right)^2 \cdot PSP_{IIN}\\
		\frac{d}{dt}PSP_{PCE} &= PSP_{t_{PCE}} \\
		\frac{d}{dt}PSP_{t_{PCE}} &= \fcolorbox{green}{white!30}{$ \frac{h_e}{\tau_e} \cdot (u + C_2 Sigm[PSP_{EIN}])  -\frac{2}{\tau_e} \cdot PSP_{t_{PCE}} - \left(\frac{1}{\tau_e}\right)^2 \cdot PSP_{PCE} $}\\
		\frac{d}{dt}PSP_{PCI} &= PSP_{t_{PCI}} \\
		\frac{d}{dt}PSP_{t_{PCI}} &= \fcolorbox{red}{white!30}{$ \frac{h_i}{\tau_i} \cdot C_4 Sigm[PSP_{IIN}])  -\frac{2}{\tau_i} \cdot PSP_{t_{PCI}} - \left(\frac{1}{\tau_i}\right)^2 \cdot PSP_{PCI} $}\\
		PSP_{PC} &= PSP_{PCE} - PSP_{PCI}\\
	\end{aligned}
\end{equation}                        



\pagebreak
%----------------------------------------------------------------------------------------
%	SECTION 1
%----------------------------------------------------------------------------------------


\section{[TEMP] Neural Mass Models}

Neural Mass Models are computational models of neural populations that aim to simulate population dynamics instead of individual neurons. They operate on the assumption that some of the most studied [frequency-based/state] effects can be modelled by treating a neural population as a single computational unit with one input and one output, representing the sum of all enclosed neuronal activity, thus greatly reducing the computational cost of the model. By combining just a few different populations in a biologically motivated structure, these models can indeed reproduce signals with realistic components. However, when using a model with this degree of simplification, one must always be aware of it's limitations. [citation-needed]

\subsection{History}

\begin{itemize}
	\item \parencite{wilson_excitatory_1972}
	\begin{itemize}
		\item Model for stuff
		\item \( x^n + y^n = z^n \cdot \sqrt{15x} \)
	\end{itemize}
	\item \parencite{lopes_da_silva_model_1974}
	\begin{itemize}
		\item Model for Alpha Activity Generation based on empirically collected physiological parameters 
		\item Histological data of thalamo-cortical relay cells (TCR) and interneurons (IN)
		\item postulated concept: TCR activate inhibitory IN, which inhibt TCRs in return, causing alpha activity
		\item DISTRIBUTED MODEL??
		\item simulation of 144 TCR and 36 IN (ratio based on histological data)
		\item simulation of depolarization and hyperpolarization
		\item Lumped Parameter Model?
		\item TODO: EXPLAIN MATHS!!!!!!!!!!!!!!
	\end{itemize}
	
	\item \parencite{lopes_da_silva_models_1976}
	\begin{itemize}
		\item motivation from freeman (1973) systematic of olfactory system
		\begin{itemize}
			\item groups and layers of with different properties
			\item aggregates: input neurons -> excitatory
			\item populations: groups of neurons with similar properties, input and output (either inhibitory or excitatory)
			\item cartels: interaction of of populations -> feedback loops
			\begin{enumerate}
				\item membrane time constant
				\item length constant
				\item gain factor (strength of interaction), probably most modifiable in terms of neural plasticity
			\end{enumerate}
		\end{itemize}
		\item distributed model (as described in Lds 1974?) "upgraded" to lumped parameter model again
		\
	\end{itemize}
\end{itemize}

\begin{itemize}
	\item \parencite{zetterberg_performance_1978}
	\begin{itemize}
		\item ...
	\end{itemize}
\end{itemize}

\begin{itemize}
	\item \parencite{jansen_neurophysiologically-based_1993-1}
	\begin{itemize}
		\item ...
	\end{itemize}
	\item \parencite{jansen_electroencephalogram_1995}
	\begin{itemize}
		\item ...
	\end{itemize}
\end{itemize}

\begin{itemize}
	\item \parencite{david_neural_2003}
	\begin{itemize}
		\item ...
	\end{itemize}
\end{itemize}

\pagebreak

\subsection{\parencite{jansen_electroencephalogram_1995}}

Jansen and Rit's model is based on the lumped parameter model by \parencite{lopes_da_silva_models_1976}. 
\begin{figure}[htp]
	\centering
	\includegraphics[width=12cm]{Figures/jansenrit/basicmodel.png}
	\caption{Simplified Model after \parencite{jansen_electroencephalogram_1995}}
	\label{Fig: Jansen Rit Simple3}
\end{figure}



\subsection{Disadvantages/Advantages}

While the abstraction that Neural Mass Models provide comes at the price of a degree of reduced biological realism, it enables a more meaningful interpretation of the displayed behavior than a highly complex detailed model that aims to simulate individual neurons and their interactions.[Coombes and Byrne 2019]
