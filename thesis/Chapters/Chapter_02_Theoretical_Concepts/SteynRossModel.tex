\section{Effects observed in the Neural-Field-Model by Steyn-Ross}\label{sec:effects-observed-in-the-neural-field-model-by-steyn-ross}
Steyn-Ross et al.~\cite{steyn_ross_modelling_2004, steyn_ross_sleep_2005, hutt_progress_2011} developed their
model of general anaesthesia based on equations by Liley at al.~\cite{liley_continuum_1999} for a mean field model of the cortex.

They demonstrated that phase-transitions in their model can account for the abrupt change
in brain-state that occurs during induction of anaesthesia~\cite{steyn_ross_modelling_2004}.
The transitions in their model occur due to the existence of multiple stable-states for certain ranges of the
parameters that model the level of anaesthetic agent in the system.
In general, they report two key-effects that resemble real EEG phenomena observed during GA:
The first is the emergence of high-amplitude oscillations during the phase-transitions
of the model.
They correlated this behavior to the \textbf{biphasic effect} occurring in
general anaesthesia (see Sec.~\ref{subsubsec:biphasic-effect}).
Second, the aforementioned effects are not ``symmetrical'',
meaning the phase-transition on the return trajectory from the ``comatose'' state is not just the reversal of the transition
during induction.
Their model predicts that the transition to the comatose state occurs at higher levels of drug concentration than
the return to the initial state.
Steyn-Ross et al.\ draw parallels to the real effect of \textbf{hysteresis},
described in Sec.~\ref{subsubsec:hysteresis}.

These findings are especially intriguing,
as the causes of observed hysteresis are yet to be definitively determined (Sec.~\ref{subsubsec:hysteresis}).
The existence of hysteresis loops in simulated models reinforces the theory of neural inertia during loss and recovery
of consciousness.

With these effects linked to the models intrinsic dynamics,
it is desirable to find the simplest possible model to further understand the prerequisites of their emergence.
In the following sections, we will delve into this experiment,
by trying to reproduce these findings with the basic JR model and the DF extension (\ref{goal:evaluate}).