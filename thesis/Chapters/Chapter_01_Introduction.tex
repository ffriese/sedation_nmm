\chapter{Introduction}\label{ch:introduction}

%\section{Initial Motivation}\label{sec:initial-motivation}
%Patients in the Minimally Conscious State can be in more or less conscious states and follow a circadian rhythm,
%with little or no external signs thereof.
%\todo{improve first sentence, cite papers to support it}
%\todo{cite papers with sleep-wake cycle in MCS \cite{cologan_sleep_2013}}
%\todo{cite papers with awareness in MCS}
%
%Lately there have been developments to establish communication channels with these patients by means of BCI-Systems.
%\todo{cite papers with communication in MCS}
%
%To successfully start BCI communication-sessions,
%it is necessary to determine the patients current state of awareness.
%While there are reliable methods to determine consciousness/awareness/wakefulness with healthy subjects,
%\todo{cite papers for EEG sleep/wake detection}
%
%The fact that signals recorded from DoC-patients are structurally different from unimpaired subjects,
%as well as the general difficulty to diagnose an unresponsive patients current state,
%pose a hard challenge.

%Riganello et al. (2021) ''Disorders of Consciousness (DOC) are a spectrum of pathologies affecting one’s ability to
%interact with the external world.
%It can be either due to a traumatic cause [1,2],
%nontraumatic cause [3,4], or a combination of both [5] and gives rise to
%ethically challenging questions [6–8], including the end-of-life decisions.
%For clinical purposes, consciousness is commonly defined by wakefulness (i.e.,
%the presence of spontaneous periods of opening the eyes) and awareness (i.e.,
%the ability for a subject to respond to the internal/external stimuli in an integrated way).
%Two possible conditions of patients with DOC are Unresponsive
%Wakefulness Syndrome/Vegetative State (UWS/VS) [9] and Minimally Conscious State (MCS) [10].
%The first is characterized by a spontaneous opening of the eyes and no sign of
%consciousness but reflexive responses to external stimuli [11,12]; the second condition
%exhibits minimal but discernible signs of non-reflex behaviours which occur
%reproducibly (yet inconsistently) as a response to visual, auditory, tactile, or noxious stimuli.''

%
%- systems for comm with wachkoma
%- problem detecting awareness (VS WAKEFULNESS!)
%- sleep/wake cylces in mcs and vs
\begin{comment}
\section{Motivation}\label{sec:motivation}

%\todo{PK-NMM, COALIA, MENON, LUPPI ->}
%
%NMMs have lately been used to simulate oscillating behavior in anesthesia~\cite{liang_pharmacokinetics-neural_2015,
%    luppi_paths_2021} as
%well as sleep and wake conditions~\cite{bensaid_coalia_2019}.
%Creating and validating realistic models greatly helps the understanding of
%the underlying processes.

Simulating the effects of sedative drugs on the human brain could help
improve our understanding of the complex dynamics of loss and recovery of consciousness.
There are several types of computational models that aim to simulate parts of the brain on different levels of detail.
While models that operate on single-neuron level should in theory be able to reproduce very
realistic brain activity, their scope is oftentimes confined to small areas of the brain. % (computational limits,
Other types of models focus on higher level structures and their interactions, while still being able to reproduce
characteristic higher-level dynamics.
In this thesis we are aiming to reproduce brain-activity in the form of an EEG-like signal,
which, due to the recording method alone, contains only information about higher-level dynamics.
Thus, the model of choice is the widely used Neural-Mass-Model.

The general neuro-chemical [?] effects of the anaesthetic drugs propofol in regard to synaptic receptors are
already well understood and provide a solid starting-point for simulation.

Building on the efforts of Liang et al.~\cite{liang_pharmacokinetics-neural_2015},
who create a model that correlates propofol infusion and resulting effect-site concentration
as well as the NMM proposed by David-Friston ~\cite{david_neural_2003}, a model to simulate a


\todo{BASE EVERYTHING ON PKK-NMM}
first step -> analyse dynamics of system

second step -> correlate to real anaesthesia

\end{comment}

\section{Motivation Deutsch}\label{sec:motivation-deutsch}
Die hochkomplexen Prozesse der Änderung von Bewusstseinszuständen im menschlichen Gehirn sind nach wie vor kaum
verstanden.
Die Definition und Feststellung von Bewusstseinszuständen beruht in der Praxis weiterhin stark auf Verhaltensbasierten
Beobachtungen (e.g. Reiz-Reaktions Tests) \textcolor{blue}{[hier möglicherweise Glasgow Coma Scale erwähnen]}.
Insbesondere bei Menschen die unter traumatischen Hirnschädigungen leiden,
ist diese Herangehensweise jedoch unzureichend, da die Gründe für ausbleibende externe Reaktionen vielseitig sein
können.
Daher werden zunehmend Messungen der Hirnaktivität durch Neuroimaging (fMRT, EEG, \dots) herangezogen um
Bewusstseinszustände
zu bestimmen und zu erforschen, sowie bestimmte Zustandsgrenzen neu zu definieren.
Dabei gibt es bereits beeindruckende Erfolge,
wie beispielsweise eine basale Kommunkation über BCI-Systeme mit Trauma-Patienten die über keinerlei äußerlich
erkennbare Regung als aufmerksam zu erkennen gewesen wären.
Die genauen Mechanismen hinter Zustandsänderungen sowie klare objektive Grenzen zwischen Zuständen sind aber
immer noch der Gegenstand von Grundlagenforschung.
Ein Ansatz um Stück für Stück bessere Einsicht in diese Prozesse zu erlangen, ist die Computer-Simulation von
kontrollierten Zustandsveränderungen, beispielsweise einer Anästhesie, um ein Verständnis für die Veränderung
von Dynamiken der Gehirnaktivtät und deren Auswirkungen zu entwickeln.
Das Betäubungsmittel Propofol wird in der Medizin verwendet um im kontrollierten Rahmen einen völligen
Bewusstseinsverlust zur Durchführung von Operationen herbeizuführen.
Der neurochemische Wirkmechanismus von Propofol auf synaptische Rezeptoren ist bereits gut
erforscht und bietet eine solide Grundlage für die Erforschung von Sedationsprozessen.
Diese Arbeit beschäftigt sich mit der Möglichkeit diese Wirkmechanismen auf Computermodelle von Neuronenpopulationen
im Gehirn anzuwenden und das entstehende Verhalten des Systems zu analysieren sowie Vergleiche zu den
bisherigen neurologischen Erkenntnissen im Bereich Anästhesie zu ziehen.
Ein gängiges Messverfahren um Anästhesie zu beobachten ist das EEG, das insbesondere wegen seines non-invasiven,
portablen und relativ kostengünstigen Einsatzes Verwendung findet.
Als Modell zur Simulation von EEG-ähnlichen Signalen bieten sich sogenannte Neural-Mass-Models (NMMs) an,
die sich auf die Modellierung der Dynamiken von größeren Neuronenpopulationen fokussieren.

%
%\section{Technical Motivation}\label{sec:technical-motivation}
%
%When developing applications that work with live EEG signals and use complex processing steps
%and/or application states,
%it is oftentimes tedious to ensure the application's correct function during development.
%This is due to multiple factors, including potential issues with the recording hardware and the
%complexity of the signal itself.
%Problems that could be caused by faulty implementations as well as the recording setup (producing poor
%signals) are especially hard to pinpoint.
%Long setup times for repeated testing sessions with real subjects further complicate resolving
%any arising issues.
%Therefore, it is very useful to quickly test the implementations with simulated data,
%which can be configured to have specific properties that are relevant to the application, ensuring
%correct function at least under `ideal' conditions.
%While this obviously does not cover the broad variety of issues that will occur in practise,
%it is immensely useful to prevent simple implementation errors and test data-flow and state progression.
%
%Naive approaches to generating EEG-like signals, such as mixing different sinoids and adding gaussian noise,
%work well for testing basic processing steps like artifact- and noise removal,
%bandpass-filtering and signal segmentation.
%However, more sophisticated algorithms, that combine high-level signal features into their reasoning
%and are oftentimes shaped to realistic data, cannot be reliably tested this way.
%
%Even more complications arise, when realistic data for the system's intended use-case is not easily available
%at development time.
%The System mentioned above \todo{actually mention the system} aims to detect the state of consciousness of patients
%with DOC, in order to select appropriate moments for communication.
%Unfortunately, the EEG signals of DOC-patients have fundamentally different characteristics compared to the signals
%recorded from healthy subjects.
%While it is relatively easy to record segments of EEG data from healthy subjects in different states of consciousness
%and label them accordingly, correctly labeled data for patients with DOC is practically unavailable for a number of
%reasons: The obvious challenge of determining the current state of consciousness for a person
%\todo{finish sentence...}

\section{Related Work}\label{sec:related-work}
There have been many efforts to simulate different states of consciousness with NMMs. A particularly sophisticated
example is Bensaid et al.'s COALIA Framework~\cite{bensaid_coalia_2019}, which is able to produce realistic
resting-state EEG-like signals for sleep and wakefulness, by using the output of over 60 interconnected NMMs and
applying a transformation onto scalp electrodes using a realistic head-model with tissue conductivities.
Liang et al.~\cite{liang_pharmacokinetics-neural_2015} focussed on relating propofol-induction to effect-site
concentration, which is an important parameter in any model for anaesthesia-simulation.

\section{Collection of Quotes}

'Although our understanding of the actions of propofol at the molecular level is quite extensive,
we do not entirely understand how these molecular effects translate into alterations in cellular,
synaptic and neural network function, and in turn cause unconsciousness [88].
This knowledge gap is, at least in part, the result of a lack of a generally accepted theory of consciousness.
In recent years, cognitive neuroscience has seen a resurgence of interest in this topic,
with attempts to integrate anaesthesia and sleep research in order to address this deficiency.
This resurgence has revealed several brain areas that play a crucial role in generation of consciousness,
and which are extensively influenced by hypnotic drugs.'
\cite{sahinovic_clinical_2018}\\[1em]


