\chapter{Motivation \& Introduction}\label{ch:introduction}




%\todo{either remove this section and call chapter `Intro \& Motivation` or add more sections}
%\section{Motivation}\label{sec:motivation}
%%%%%%%%%%%%%%%%%%%%%%%%%%%%%%%%%%%%%%%%%%%%%%%%%%%%%%%%%%%%%%%
% How do states of consciousness change? How are they defined %
%%%%%%%%%%%%%%%%%%%%%%%%%%%%%%%%%%%%%%%%%%%%%%%%%%%%%%%%%%%%%%%
The highly complex processes of changing states of consciousness in the human brain are still barely understood.
States of consciousness have historically been defined based on behavioral observations such as responsiveness to
stimuli.
As no other indications, save simple physiological measurements such as pulse and respiration rate
were available, the concept of awareness was tightly linked to observable behavior.
While the great majority of medical applications to determine levels of consciousness,
even in practical clinical contexts~\cite{jain_glasgow_2022},
are still covered by these observations,
there are cases in which the link between consciousness and displayed behavior falls apart;
striking examples are some disorders of consciousness (DOC) as well as the total locked-in
syndrome~\cite{bauer_varieties_1979},
where patients are unable to display any visible behavior,
while maintaining some or even full awareness.
Although cases like these are rare, they highlight that consciousness needs to be studied at its source -- on the
level of brain-activity -- to fully understand the mechanisms that lead to its different states.
%
% Additionally, some edge-cases surrounding DOCs may require changes to common state-definitions,
% that are based on [...]
%
%%%%%%%%%%%%%%%%%%%%%%%%%%%%%%%%%%%%%%
% How can we measure brain-activity? %
%%%%%%%%%%%%%%%%%%%%%%%%%%%%%%%%%%%%%%
Brain-activity can be measured directly or indirectly.
Direct measurement with electrodes in the brain, while more accurate,
poses the obvious problems of its invasive nature.
Various neuroimaging techniques, which indirectly measure brain-activity, have been used to study levels of
awareness since their emergence.
Already in the 1880`s, Angelo Mosso measured scalp-pulse variations in subjects
with skull-injuries during challenging tasks,
and concluded an increased blood-flow to the brain~\cite{mosso_ueber_1881}.
Improved techniques like the fMRI and the EEG allow for non-invasive observation of brain-activity and have been
used extensively to study the dynamics of the brain during sleep and loss and return of consciousness
(e.g., ~\cite{murphy_propofol_2011, lee_classification_2017, kuizenga_biphasic_2001, boly_connectivity_2012}
and many others).
Some phenomena, like predictable changes in signal-frequencies and some critical brain regions involved in
consciousness-modulation, e.g., to induce sleep,
have been identified and are well documented~\cite{tantirigama_perspective_2020, wu_neuroscience_2018}.
Techniques to measure levels of consciousness via neuroimaging have been proposed~\cite{sigl_introduction_1994,
    casali_theoretically_2013} and used in practice~\cite{mathur_bispectral_2022}.
However, exact mechanisms behind state-changes are still object of fundamental research~\cite{tantirigama_perspective_2020}.

%%%%%%%%%%%%%%%%%%%%%%%%%%%%%%%%%%%%%%%%%%%%%%%
% How to study state-changes most efficently? %
%%%%%%%%%%%%%%%%%%%%%%%%%%%%%%%%%%%%%%%%%%%%%%%
% controlled conditions -> anaesthesia
Controlled state-changes, like inducing loss-of-consciousness with anaesthetic drugs,
are exceptionally well-suited for studying the underlying mechanisms of consciousness,
as they allow for repeatable conditions~\cite{bonhomme_general_2019}.
% propofol as a well-understood agent
Furthermore, the neuro-chemical mechanisms of the drugs provide a good starting-point for modeling these processes.
The sedative propofol is the most commonly used drug to induce controlled loss of consciousness
during general anaesthesia (GA)~\cite{miner_clinical_2007}.
Propofol's neuro-chemical mode of action and its specific effects on synaptic receptors
are well understood~\cite{sahinovic_clinical_2018, kitamura_effects_2003},
which lays a solid foundation to study the brain-dynamics associated with loss of consciousness.
When it comes to monitoring the brain-activity during anaesthesia,
the EEG stands out due to its practicability, low cost, and high temporal resolution.
As a result, there is already a vast body of research in this area.
Studying the origins and mechanisms behind the measured activity remains immensely challenging nonetheless,
largely due to the highly complex interconnected brain itself.

A promising approach to further our insights step by step, is to simulate state-changes
like anaesthesia in computer-models to develop an understanding of the dynamics of brain-activity and its
effects on consciousness. %% controlled induction/loss of consciousness to directly test hypothesis experimentally
Simulations enable testing of specific hypotheses by modelling the proposed conditions and analysing the outcome.
Computer-models of the brain (or parts of it) exist on different levels~\cite{panahi_generative_2021},
including simulations of individual neurons and population-based approaches.
Neural-Mass-Models (NMMs), members of the latter, model synaptic connections between populations of different types
of neurons and have proven to replicate multiple abstract phenomena of brain-dynamics~\cite{bojak_neural_2014,
    knösche_jansen-rit_2014} within a small cortical column.
The simplifications they provide allow efficient real-time simulation of signals,
while preserving many important characteristics of EEG or MEG signals.
Population models generate output resembling EEG recordings,
as they are designed to operate on the same level of abstraction
from the underlying single-cell-activity~\cite{deco_dynamic_2008}.
While a computationally highly expensive model that simulates all neurons of a population individually might
be technically more accurate,
it can be argued that it would add unnecessary complexity
when studying the abstract phenomena observable in the EEG.
Additionally, population models can easily be scaled to whole-brain models
that accomplish near-real-time performance on conventional hardware (e.g,~\cite{bensaid_coalia_2019}).
Simulating every neuron in the brain on the other hand, is an enormous task,
which to date even supercomputers are unable to solve (although it is actively pursued)~\cite{jordan_extremely_2018}.

By combining the well-known properties of propofol with an NMM, we can expect to gain insights into the mechanisms that
govern changes of brain-dynamics in the presence of consciousness-altering drugs.
While the abstract nature of NMMs, along with its limitations must always be
considered~\cite{deschle_validity_2021},
the results of simulated experiments can provide valuable research indications.
%\todo{start section `contribution of this thesis:` to lead into the following}
\section{Related Work \& Goals}\label{sec:related-work-and-goals}
%%%%%%%%%%%%%%%%%%%%%%%%%%%%%%%%%%%
% MAYBE STOP HERE, REWRITE CONTRIBUTION AND GOALS PART, MOVE OTHER TEXT TO NMM INTRO
%%%%%%%%%%%%%%%%%%%%%%%%%%%%%%%%%%%
One of the most commonly used models in the area of neural population models is the
Jansen-Rit~\cite{jansen_electroencephalogram_1995} (JR) NMM,
which is based on efforts by Wilson \& Cowan~\cite{wilson_excitatory_1972},
Lopes da Silva et al.~\cite{lopes_da_silva_model_1974, lopes_da_silva_models_1976} and
Zetterberg et al.~\cite{zetterberg_performance_1978}.
It is one of the most simple and basic population models,
while retaining `a considerable degree of biological realism` and
`producing a surprisingly rich repertoire of dynamic behaviors`~\cite{knösche_jansen-rit_2014}.
~\cite{spiegler_bifurcation_2010}.
Many efforts in the field are based on the JR NMM (e.g.,~\cite{wendling_relevance_2000, david_neural_2003,
    moran_dynamic_2009, cona_thalamo-cortical_2014, bensaid_coalia_2019} and others).

Neural-Mass models have been widely used to simulate the properties of different states of
consciousness.
Cona et al.~\cite{cona_thalamo-cortical_2014} adapted multiple interconnected JR NMM modules to simulate effects
of different stages of sleep.
Their model accounts for thalamo-cortical modulation of cortico-cortical connectivity as an important mechanism to
influence these transitions.
Another sophisticated example is Bensaid et al.'s COALIA Framework~\cite{wendling_epileptic_2002, bensaid_coalia_2019},
which makes use of a similar connectivity structure, while using over 60 interconnected NMM-modules and applying a
transformation onto scalp electrodes using a realistic head-model with tissue conductivities.
The resulting system is able to simulate the influence of Transcranial Magnetic Stimulation (TMS) during either
a waking or sleeping state.

% steyn_ross_sleep_2005
However, since we are looking to simulate propofol-induced unconsciousness, other approaches bear special
consideration:
To simulate phenomena specific to general anaesthesia,
Steyn-Ross et al.~\cite{steyn_ross_modelling_2004, hutt_progress_2011} developed a neural field model based
on Liley et al.'s continuum model~\cite{liley_continuum_1999}.
Neural field models are similar to neural mass models,
but also take into account the continuity of the cortex and the spatial distribution of
neurons~\cite{glomb_computational_2022}, adding complexity to the model.

%\section{Goal}\label{sec:goal}
The JR model has its greatest advantages in its simplicity (even including extensions) and serves as 'Occam's Razor'
for many applications in the field~\cite{kuhlmann_neural_2016}.
As its fundamental structures have proven to be a working local basis for sophisticated approaches as the COALIA
framework,
it appears desirable to evaluate its ability to simulate the effects of anaesthesia and serve as a simple tool to
further consciousness research in that area.

This work aims to reproduce the behavior observed in Steyn-Ross et al.'s neural field model~\cite{hutt_progress_2011}
using a neural mass approach based on the JR model,
thus helping to investigate the prerequisites for the emergence of the phenomena in brain-dynamics typical to
anaesthesia -- specifically the existence of propofol hysteresis
and a biphasic effect as reported by~\cite{hutt_progress_2011}.
While the basic JR model is able to produce rich patterns of dynamic behavior,
large regions of its parameter-space generate sinusoidal signals with a single pronounced frequency by design,
which might limit its ability to reproduce the expected phenomena.
Therefore, the subpopulation-extension proposed by David \& Friston~\cite{david_neural_2003} is also employed to
produce a more realistic baseline frequency spectrum and provide a broader parameter space.

\vspace{1em}
\noindent In summary, this thesis aims to accomplish the following:
\begin{enumerate}[label=\textbf{Goal \arabic*}:, ref=\textbf{Goal~\arabic*}, align=left, leftmargin=*]
    \item Implement the basic Jansen-Rit model and validate for correct output\label{goal:jr_model}
    \item Extend the model by adding subpopulations as proposed by David-Friston and validate the implementation\label{goal:df_model}
    \item Include a parameterizable representation of the effects of propofol in the model \label{goal:implement_propofol}
    \item Simulate a session of general anaesthesia with the implemented models \label{goal:sim_ga}
    \item Evaluate if either or both of the models can reproduce the effects described by Steyn-Ross et al. \label{goal:evaluate}
\end{enumerate}



