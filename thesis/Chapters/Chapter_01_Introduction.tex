\chapter{Introduction}\label{ch:introduction}





\section{Motivation}\label{sec:motivation}
%%%%%%%%%%%%%%%%%%%%%%%%%%%%%%%%%%%%%%%%%%%%%%%%%%%%%%%%%%%%%%%
% How do states of consciousness change? How are they defined %
%%%%%%%%%%%%%%%%%%%%%%%%%%%%%%%%%%%%%%%%%%%%%%%%%%%%%%%%%%%%%%%
The highly complex processes of changing states of consciousness in the human brain are still barely understood.
States of consciousness have historically been defined based on behavioral observations such as responsiveness to
stimuli.
As no other indications, save simple physiological measurements such as pulse and respiration rate,
were available, the concept of awareness was tightly linked to observable behavior.
While the great majority of medical applications to determine levels of consciousness,
even in practical clinical contexts~\cite{jain_glasgow_2022},
are still covered by these observations,
there are cases in which the link between consciousness and displayed behavior falls apart;
striking examples are some disorders of consciousness (DOC) as well as the total locked-in
syndrome~\cite{bauer_varieties_1979},
where patients are unable to display any visible behavior,
while maintaining some or even full awareness.
Although cases like these are rare, they highlight that consciousness needs to be studied at its source -- on the
level of brain-activity -- to fully understand the mechanisms that lead to its different states.
%
% Additionally, some edge-cases surrounding DOCs may require changes to common state-definitions,
% that are based on [...]
%
%%%%%%%%%%%%%%%%%%%%%%%%%%%%%%%%%%%%%%
% How can we measure brain-activity? %
%%%%%%%%%%%%%%%%%%%%%%%%%%%%%%%%%%%%%%
Brain-activity can be measured directly or indirectly.
Direct measurement with electrodes in the brain, while more accurate,
poses the obvious problem of its invasive nature and faces the challenge of realistically\todo{rephrase} only allowing very
localized measurements \textcolor{blue}{[citation needed]}.
Various neuroimaging techniques, which indirectly measure brain-activity, have been used to study levels of
awareness since their emergence.
Already in the 1880`s, Angelo Mosso measured scalp-pulse variations in subjects
with skull-injuries during challenging tasks,
and concluded an increased blood-flow to the brain~\cite{mosso_ueber_1881}.
Improved techniques like the fMRI and the EEG allow for non-invasive observation of brain-activity and have been
used extensively to study the dynamics of the brain during sleep and loss and return of consciousness
\textcolor{blue}{[citation needed]}.
Some phenomena, like predictable changes in signal-frequencies and some critical brain regions involved in
consciousness-modulation have been identified and are well documented \textcolor{blue}{[citation needed]}.
Techniques to measure levels of consciousness via neuroimaging have been proposed~\cite{sigl_introduction_1994,
    casali_theoretically_2013} and used in practise~\cite{mathur_bispectral_2022}.
However, exact mechanisms behind state-changes are still object of fundamental research.

%%%%%%%%%%%%%%%%%%%%%%%%%%%%%%%%%%%%%%%%%%%%%%%
% How to study state-changes most efficently? %
%%%%%%%%%%%%%%%%%%%%%%%%%%%%%%%%%%%%%%%%%%%%%%%
Controlled state-changes, like inducing loss-of-consciousness with anaesthetic drugs,
are exceptionally well-suited for studying the underlying mechanisms of consciousness,
as they allow for repeatable conditions.
Furthermore, the neuro-chemical mechanisms of the drugs provide a good starting-point for modeling these processes.
The sedative propofol is the most commonly used drug to induce controlled loss of consciousness
during general anaesthesia.
Propofol's neuro-chemical mode of action and its specific effects on synaptic receptors are well understood,
which lays a solid foundation to study the brain-dynamics associated with loss of consciousness.

A promising approach to further our insights step by step, is to simulate controlled state-changes
(such as anaesthesia) in computer-models to develop an understanding of the dynamics of brain-activity and its
effects on consciousness. %% controlled induction/loss of consciousness to directly test hypothesis experimentally
Computer-models of the brain (or parts of it) exist on different levels~\cite{panahi_generative_2021},
including simulations of individual neurons and population-based approaches.
Neural-Mass-Models (NMMs), members of the latter, model synaptic connections between populations of different types
of neurons and have proven to replicate multiple abstract phenomena of brain-dynamics\cite{bojak_neural_2014,
    knösche_jansen-rit_2014} within a small cortical
column.
The simplifications they provide allow efficient real-time simulation of signals,
while preserving many important characteristics of EEG or MEG signals.

\todo{improve explanation of NMMs}

By combining the well-known properties of propofol with an NMM, we can hope to gain insights into the mechanisms that
govern changes of brain-dynamics in the presence of consciousness-altering drugs.
While the abstract nature of NMMs, along with its limitations must always be
considered~\cite{deschle_validity_2021},
the results of simulated experiments can provide valuable research indications.


One of the most commonly used models in the area of neural population models is the
Jansen-Rit~\cite{jansen_electroencephalogram_1995} (JR) NMM,
which is based on efforts by Wilson \& Cowan~\cite{wilson_excitatory_1972},
Lopes da Silva et al.~\cite{lopes_da_silva_model_1974, lopes_da_silva_models_1976} and
Zetterberg et al.~\cite{zetterberg_performance_1978}.
It is one of the most simple and basic population models,
while retaining `a considerable degree of biological realism` and
`producing a surprisingly rich repertoire of dynamic behaviors`~\cite{knösche_jansen-rit_2014}.
Many efforts in the field are based on the JR NMM, e.g.~\cite{wendling_relevance_2000, david_neural_2003,
    moran_dynamic_2009, spiegler_bifurcation_2010, cona_thalamo-cortical_2014, bensaid_coalia_2019} and others.



Neural-Mass models have been widely used to simulate the properties of different states of consciousness.
Cona et al.~\cite{cona_thalamo-cortical_2014} adapted the JR NMM to simulate effects
of different stages of sleep.
Their model accounts for thalamo-cortical modulation of cortico-cortical connectivity as an important mechanism to
influence these transitions.
Bensaid et al.'s COALIA Framework~\cite{bensaid_coalia_2019},
makes use of a similar structure, while using over 60 interconnected NMM-modules and applying a
transformation onto scalp electrodes using a realistic head-model with tissue conductivities.

% steyn_ross_sleep_2005
As we are looking to simulate propofol-induced unconsciousness, similar approaches bear special consideration:
To simulate phenomena specific to general anaesthesia,
Steyn-Ross et al.~\cite{steyn_ross_modelling_2004, hutt_progress_2011} developed a model based
on Liley et al.'s continuum model~\cite{liley_continuum_1999}.
\todo{short explanation of continuous cortical field theories CFT, the difference to `normal' NMMs}

This work aims to use a non-CFT approach based on the JR model and evaluate if it is able to reproduce the
behavior observed in Steyn-Ross et al.'s CFT model ~\cite{hutt_progress_2011}.
While the basic JR model is able to produce rich patterns of dynamic behavior~\cite{spiegler_bifurcation_2010},
large regions of its parameter-space generate sinusoidal signals with a single pronounced frequency,
which might limit its ability to reproduce the expected phenomena (as noted by~\cite{kuhlmann_neural_2016}).
Therefore, the subpopulation-extension proposed by David \& Friston~\cite{david_neural_2003} is also employed to
produce a more realistic baseline frequency spectrum.

%\cite{luppi_paths_2021}}
%
%Liang et al.~\cite{liang_pharmacokinetics-neural_2015} focussed on relating propofol-induction to effect-site
%concentration, which is an important parameter in any model for anaesthesia-simulation.

%
%\section{Collection of Quotes}
%
%'Although our understanding of the actions of propofol at the molecular level is quite extensive,
%we do not entirely understand how these molecular effects translate into alterations in cellular,
%synaptic and neural network function, and in turn cause unconsciousness [88].
%This knowledge gap is, at least in part, the result of a lack of a generally accepted theory of consciousness.
%In recent years, cognitive neuroscience has seen a resurgence of interest in this topic,
%with attempts to integrate anaesthesia and sleep research in order to address this deficiency.
%This resurgence has revealed several brain areas that play a crucial role in generation of consciousness,
%and which are extensively influenced by hypnotic drugs.'
%\cite{sahinovic_clinical_2018}\\[1em]


