\chapter{Discussion}
\incomplete{i'm not happy with this section yet...}
%% goal / hypotheses
The general goal of this thesis was to validate whether the basic single-column Jansen-Rit model would be able to
reproduce the main phenomena observed in~\cite{hutt_progress_2011}.
That is a biphasic effect and a pronounced drug-hysteresis.
Given it's limited ability to output realistic frequency distributions,
and the hypothesis that this would hinder those efforts,
the extension of the David-Friston model which overcomes these limitations was also employed to reach this goal.

%% results
Interestingly, even the basic single-column JR Model produces a strong bi-phasic response and a minor hysteresis effect
can be observed, as the transitions in to and out of the `unstable' state are not perfectly `symmetrical' when raising
versus decreasing the simulated propofol effect.
The biphasic effect occurs very quickly after increasing the IPSP duration, which may in part be due to the default
parameters lying close to a region that produces what David and Friston call `hypersignals`.
The resulting spectrogram does not have many attributes of a realistic EEG signal during anaesthesia,
as was to be expected due to the aforementioned limitations.
Generally, the results do suggest that the basic JR model has intrinsic properties that produce the same general
effects that Steyn-Ross et al.\ reported in their neural field model~\cite{hutt_progress_2011}.

% context of results (Steyn-Ross, Purdon, ...)


% comparison to other approaches (COALIA / Kuhlmann)

Kuhlmann et al.'s efforts to use the basic JR model for depth-of-anaesthesia tracking did produce encouraging results,
even though they concluded that the model might be too simple for their approach.
They experimented with a slight modification of adding self-inhibition to the IIN population without notable success.
It would be interesting to validate this method with the subpopulation extension used in this thesis,
as is appears to be able to model a rich palette of realistic signal states and frequency distributions.


% LIMITS OF THE MODEL!!!!
While the results of the system are promising, it must always be kept in mind that the spatial abstractions of a
population model are not the only simplification in the system used in this thesis.
Many key-concepts and parameters of the model are strictly rooted in neuro-biological evidence,
but there has also been parameter tuning to match desired outputs (e.g.\ the choice of weights for the subpopulations
in~\cite{david_neural_2003} was ultimatively motivated by the resulting frequency-spectra).
The resulting weights do lie well within the plausible ranges [citation-needed],
but should not be considered given.
Additionally, studying the generated signal of a single macro-column cannot be used to make dependable
general statements about processes as complex as the loss and emergence of consciousness.

Some of these concerns can be addressed by WHOLE BRAIN NMM MODELS,
which consist of many of these columns, ideally spatially realistically interconnected and with propagation delays
between regions that are further apart.




\section{Outlook}
\todo{How could the model be extended: e.g. more realism by incorporation into a whole-brain system like COALIA, ...}
\todo{Possible further Applications: Depth-Of-Anaesthesia-Tracking ~\cite{kuhlmann_neural_2016}, ...}