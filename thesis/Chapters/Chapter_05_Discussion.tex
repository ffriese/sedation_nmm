\chapter{Discussion}

While the NMM used for simulation is a very rough abstraction of cortical dynamics,
multiple parallels to effects observed during GA can be drawn:

\begin{enumerate}

    \item Two, distinguishable \textbf{stable states} can be observed.
    The frequency changes bear strong similarities to the switch to unconsciousness in GA
    \cite{purdon_electroencephalogram_2013, ni_mhuircheartaigh_slow_wave_2013}.
    \item Induction and Emergence are asymmetrical (\textbf{hysteresis}).
    The system predicts that the state-change `LOC` occurs at higher concentrations than `ROC`.

    \item During state transitions, there is a strong \textbf{biphasic effect}.
    The system predicts that the frequency range below $\SI{25}{\hertz}$ receives a temporary amplitude boost during
    the phase transitions, which disappears while the parameter changes continue in the same direction.
    \item In the `unconscious` state, frequency distribution stabilizes independent of further increasing decay-time
    (\textbf{slow-wave-activity saturation}?~\cite{ni_mhuircheartaigh_slow_wave_2013}).

\end{enumerate}

\todo{elaborate on all of the above...}
\todo{draw parallels to~\cite{hutt_progress_2011}, which discusses mainly the same effects,
    although with a different NMM (or `mean-field-model`)...}

\section{Outlook}
\todo{How could the model be extended}