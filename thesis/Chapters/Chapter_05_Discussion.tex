\chapter{Discussion}

The shared objective of Steyn-Ross et al.\ and this thesis
is to further the understanding of loss and emergence of consciousness
by capturing anaesthetic effects in an abstract model of brain activity.
Steyn-Ross at al.\ were able to show
that their model produces hysteresis and a biphasic effect~\cite{hutt_progress_2011}.
The general question of this thesis was,
whether there are simpler models
that are able to reproduce these phenomena (\ref{goal:evaluate}).

The first and arguably most obvious candidate for this was the widely used Jansen-Rit Model,
which is well known for a wide range of dynamics, despite its simplicity.
Generally, the results do suggest that the basic JR model has intrinsic properties that produce
effects similar to those reported in the neural field model~\cite{hutt_progress_2011}:
The default configuration of the basic neural-mass model~\cite{jansen_electroencephalogram_1995}
produces a strong biphasic response (see Fig.~\ref{fig:sedation_sim_jr}),
when increasing the inhibitory decay-time constant.
The same effect can be observed on the return path.
There appears to be little difference in the spectral composition of the two instances of high-amplitude activity.
A minor hysteresis of the model can also be observed, as the transitions in and out of the `unconscious' state
occur at slightly different concentrations.
In the JR model, the biphasic effect occurs very quickly after increasing the IPSP duration, which may in part be due
to the vicinity of the default values for $\tau_e$ and $\tau_i$ to a parameter-region that produces what David and Friston call
`hypersignals`~\cite{david_neural_2003}.
In fact, their exploration of different synaptic kinetics~\cite[Fig. 4]{david_neural_2003}
is not unlike our modulation of the inhibitory decay-time,
but differs in the detail that they simultaneously adjust the PSP amplitude to enforce oscillatory behavior.
This appears to prevent their exploration from encountering a biphasic effect.

The lack of the basic model's inability to output realistic frequency distributions or a pronounced hysteresis
has motivated the next tier of model sophistication, namely the extension of David \&
Friston~\cite{david_neural_2003}.
Their contribution of splitting the populations into subpopulations with different PSP properties
enables the model to produce signals with more realistic frequency distributions.
Indeed, the spectra generated during simulated sedation sessions with the default David-Friston model
(see Fig.~\ref{fig:sedation_sim_df}) bear similarities to those observed in
real EEG recordings during general anaesthesia~\cite{purdon_electroencephalogram_2013}.
The biphasic effect occurring in the DF model is slightly less prominent than its counterpart in the JR model
(just like the signal in general lacks the higher amplitude of the basic model)
but maintains all relevant characteristics.
The two occurrences for induction and emergence strong resemble each other in spectral composition,
with no noticeable differences for similar $\lambda$ values.
During emergence, the transition state prevails through lower values of $\lambda$,
continuing the trend of linearly changing frequency composition.
Importantly for our hypothesis,
this means that the extended model also produces a clearly visible counter-clockwise hysteresis,
with distinctly different start and end-concentrations for induction and emergence.

When comparing the decay-time factor for "LOC" and "ROC" in the DF model
with real anaesthesia (see Sec.~\ref{subsubsec:realistic-prop-conc-during-ga}),
the reported $10-\SI{11}{\micro\molar}$ for LOC and ROC,
which approximately double the IPSP decay-time,
provide a good general match to the $\lambda$ values close to $2$.
Another interesting observation, is that in both models during the `unconscious` state,
the frequency distribution stabilizes independent of further increasing decay-time.
This invites a comparison with the real world effect of slow-wave-activity
saturation~\cite{ni_mhuircheartaigh_slow_wave_2013}.
The analysis of common EEG frequency bands (see Appendix~\ref{ch:bandpower_plots}) reveals
some expected developments, e.g., in the baseline-state, alpha (and beta in the DF-model) dominates,
while in the `unconscious` state there is a prevalence of delta activity.
This is especially visible in the relative band-power plots (Fig.~\ref{fig:rel_bands}).

The fact that relatively simple computational models can predict the general frequency dynamics occurring during the
drug-induced loss of consciousness,
invites the speculation that effects such as hysteresis and the biphasic effect observed in real EEG recordings can at
least in part be attributed to universally inherent brain-mechanics on population level.
While the effects observed during these simulations provide little to no specific information about the factors
involved in the emergence of consciousness,
they suggest that abstract population dynamics might play a crucial role in modulating the necessary preconditions.

%%% LIMITS
Although the results of the system are enticing,
there are clear limits to its realism.
It must always be kept in mind that the spatial abstractions of a
population model are not the only simplification in the system used in this thesis.
Many key-concepts and parameters of the model are strictly rooted in neuro-biological evidence,
but there has also been parameter tuning to match desired outputs -- e.g., the choice of time-constants and weights for
the subpopulations by David \& Friston was ultimately motivated by the
resulting frequency-spectra~\cite{david_neural_2003},
while Jansen \& Rit modified the connectivity parameter $C$ to produce
alpha-activity~\cite{jansen_electroencephalogram_1995}.
The chosen values do lie well within biologically plausible ranges~\cite{gulledge_synaptic_2005},
but should not be considered a given just because they cause "realistic" output.
Obviously, studying the generated signal of a single macro-column cannot be used to make dependable
general statements about processes as complex as the loss and emergence of consciousness.
The breakdown of stability in small columns is, however, a good indication for the loss of necessary
conditions for consciousness in the system as a whole.
Despite severe limits,
it is remarkable that the simple model can capture many relevant aspects of anaesthesia dynamics.
The goal of this thesis was to find a simplistic abstraction that can nonetheless reproduce specific phenomena
associated with consciousness modulation.
Improving realism while preserving abstraction is naturally still a desirable extension.

Some of the concerns raised above are addressed in approaches like COALIA~\cite{bensaid_coalia_2019},
where different neuron-type kinetics are modelled in a way that is similar to the subpopulation approach,
but appears to be more biologically grounded.
Additionally, their system connects dozens of individual NMMs into a "whole-brain" network-model,
that even includes a dedicated thalamic module with specific kinetics.
The interconnections are based on real brain-connectivity data and account for propagation delays
between regions that are further apart.

\section{Outlook}
Extending the system into a whole-brain model like COALIA is an obvious step towards realism and could be used to
study anaesthesia dynamics in further detail.
But even a smaller step in that direction,
like extending the model to use more than one macro-column could be an improvement and allow exploring other
dynamic effects of anaesthesia (e.g., burst-supression~\cite{bojak_emergence_2015}).

However, the development of models that aim to simulate anaesthesia is not only valuable for consciousness research.
Kuhlmann et al.'s~\cite{kuhlmann_neural_2016} efforts to use the JR model for depth-of-anaesthesia tracking
by estimating the model parameters for a given real EEG signal produced encouraging results,
even though they concluded that the basic JR model might be too simple for their approach.
They experimented with a slight modification of adding self-inhibition to the IIN population without notable success.
It would be interesting to validate this method with the subpopulation extension used in this thesis.
An application like the depth-of-anaesthesia tracking that Kuhlmann et al.\ proposed
could estimate multiple clinically relevant physiological variables like the PSP amplitude and rate constant.
When matured, such a system could have advantages that would set it apart from current methods,
which rely on algorithmic evaluation of extracted EEG features.
In practice, anaesthetic agents like propofol are often administered in combination with analgesic (pain-reducing)
agents (e.g., remifentanil).
This substantially lowers the concentration requirements for the anaesthetic agent to achieve the same level of
sedative effect,
while also helping to increase the quality of anaesthesia, stabilizing the blood pressure and improving recovery
\cite{chen_propofol_2021}.
Additionally, different anaesthetics have different modes of action that
lead to unconsciousness~\cite{tantirigama_perspective_2020},
and produce different EEG signatures.
This may render current approaches like the Bispectral Index~\cite{mathur_bispectral_2022}
unreliable,
as its single (and proprietary) evaluation algorithm based on predetermined EEG features
may struggle to regard the effects of different agents~\cite{avidan_prevention_2011}
or disentangle interactions of multiple drugs~\cite{kuhlmann_neural_2016}.
Incorporating knowledge about the modes of action of specific drugs into a population model
and estimating these parameters might promise more nuanced evaluation of brain states.
Being able to model and estimate physiological variables during general anaesthesia
could be a valuable tool for clinical practice in general.

\section{Conclusion}
This work set out to evaluate if simple neural-mass models can reproduce the hysteresis and biphasic effect
that is observed during general anaesthesia,
emulating the results of a more complex neural-field model by Steyn-Ross et al.
We demonstrated that the Jansen-Rit model,
a popular and very basic neural-mass model designed to generate alpha-activity,
produces a biphasic effect and a minor but hardly noticeable hysteresis.
Extending the model with a simple subpopulation concept proposed by David \& Friston, enables the model to produce
not only more realistic baseline activity, but also a clearly pronounced hysteresis in addition to the biphasic effect.
These results suggest that some aspects of these dynamics might be inherently rooted
within population-level mechanics in the brain.
They also reinforce the notion that the observed hysteresis of sedatives like
propofol is not exclusively caused by inaccurate pharmacokinetic models but also by the existence of neural inertia.
