\chapter{Discussion}
%\todo{-general structure:\\
%-common goals of steyn-ross and this thesis:\\
%--- capture anaesthetic effect with simple(?) model
%-steyn-ross as "landmark" effects\\
%-own contribution:\\
%--- given the steyn ross model -> to what extend can simpler models capture this\\
%--- first candidate: JR\\
%----- biphasic \& minimal hysteresis\\
%----- limits: ...(maybe kuhlmann reference)\\
%--- this motivated next tier of model-sophistication: DF\\
%----- ...}
The shared objective of Steyn-Ross et al. and this thesis,
is to further understanding of the loss and emergence of consciousness,
by capturing anaesthetic effects in an abstract model of brain activity.
Steyn-Ross at al.\ were able to show,
that their model produces hysteresis and a biphasic effect~\cite{hutt_progress_2011}.
The general question of this thesis was,
whether there are simpler models,
that are able to reproduce these phenomena.
The first and arguably most obvious candidate for this was the widely used Jansen-Rit Model,
which is well known for a wide range of dynamics, despite its simplicity.
Generally, the results do suggest that the basic JR model has intrinsic properties that produce
effects similar to those reported in the neural field model~\cite{hutt_progress_2011}:
the default configuration of the famous 1995 model~\cite{jansen_electroencephalogram_1995}
produces a strong biphasic response (see Fig.~\ref{fig:sedation_sim_jr}),
when increasing the inhibitory decay-time constant.
The same effect can be observed on the return path.
There appears to be little difference in the spectral composition of the two instances of high-amplitude activity.
A minor hysteresis of the model can also be observed, as the transitions in and out of the `unconscious' state are not
perfectly `symmetrical' \iltodo{quantify}.
Another interesting observation is that in the `unconscious` state, the frequency distribution stabilizes independent
of further increasing decay-time.
This invites a comparison with the real world effect of  slow-wave-activity
saturation ~\cite{ni_mhuircheartaigh_slow_wave_2013}.
In the JR model, the biphasic effect occurs very quickly after increasing the IPSP duration, which may in part be due
to the default parameters lying close to a region that produces what David and Friston call
`hypersignals`\cite{david_neural_2003}.
\todo{maybe mention kuhlmann here, their criticism of JR limits}

Given the basic model's inability to output realistic frequency distributions or a pronounced hysteresis,
this motivated the next tier of model sophistication, namely the extension of David \&
Friston\cite{david_neural_2003}.
Their contribution of splitting the populations into subpopulations with different PSP properties,
enables the model to produce signals with more realistic frequency distributions.
Indeed, the spectra generated during simulated sedation sessions with the default David-Friston model
(see Fig .\ref{fig:sedation_sim_df}) bear similarities to those observed in
real EEG recordings during general anaesthesia~\cite{purdon_electroencephalogram_2013}.
\todo{describe spectral features}
The biphasic effect occurring in the DF model is slightly less prominent than its counterpart in the JR model,
just like the signal in general lacks the higher amplitude of the basic model,
but maintains all relevant characteristics.
\todo{describe difference (or lack thereof) between spectral features of LOC and ROC}

More importantly for our hypothesis however,
the extended model also produces a clearly visible counter-clockwise hysteresis.

The fact that relatively simple computational models can predict the general frequency dynamics occurring during the
drug-induced loss of consciousness,
invites the speculation that effects such as hysteresis and the biphasic effect observed in real EEG recordings can at
least in part be attributed to universally inherent brain-mechanics on population level.
While the effects observed during these simulations provide little to no specific information about the factors
involved in the emergence of consciousness,
they suggest that abstract population dynamics might play a crucial role in modulating the necessary preconditions.

Although the results of the system are enticing,
there are clear limits to its realism.
It must always be kept in mind that the spatial abstractions of a
population model are not the only simplification in the system used in this thesis.
Many key-concepts and parameters of the model are strictly rooted in neuro-biological evidence,
but there has also been parameter tuning to match desired outputs (e.g., the choice of time-constants and weights for
the subpopulations in~\cite{david_neural_2003} was ultimately motivated by the resulting frequency-spectra).
The chosen values do lie well within biologically plausible ranges \citationneeded,
but should not be considered a given just because they cause "realistic" output.
Obviously, studying the generated signal of a single macro-column cannot be used to make dependable
general statements about processes as complex as the loss and emergence of consciousness.
The breakdown of stability in small columns is however a good indication for the loss of necessary
conditions for consciousness in the system as a whole.
Despite severe limits,
it is remarkable that the simple model can capture many relevant aspects of anaesthesia dynamics.
The goal of this thesis was to find a simplistic abstraction that can nonetheless reproduce specific phenomena
associated with consciousness modulation.
Improving realism, while preserving abstraction is naturally still a desirable extension.

\section{Outlook}
%Modeling the mechanisms related to consciousness remains a challenging task
\todo{possible extension: modeling further dynamic effects of anaesthesia? (burst-supression, ...)}
\todo{How could the model be extended: e.g., more realism by incorporation into a whole-brain system like COALIA, ...}
Some of the concerns raised above are addressed in approaches like COALIA~\cite{bensaid_coalia_2019},
where different neuron-type kinetics are modelled in a way that is similar to the subpopulation approach,
but appears to be more biologically grounded.
Additionally, their system connects dozens of individual NMMs into a "whole-brain" network-model,
that even includes a dedicated thalamic module with specific kinetics.
The interconnections are based on real brain-connectivity data and account for propagation delays
between regions that are further apart.

The development of models that aim to simulate anaesthesia is not only valuable for consciousness research.
\todo{describe kuhlmann paper in more detail}
Kuhlmann et al.'s~\cite{kuhlmann_neural_2016} efforts to use the basic JR model for depth-of-anaesthesia tracking did
produce encouraging results,
even though they concluded that the model might be too simple for their approach.
They experimented with a slight modification of adding self-inhibition to the IIN population without notable success.
It would be interesting to validate this method with the subpopulation extension used in this thesis.
An application like the Depth-Of-Anaesthesia-Tracking that Kuhlmann et al. proposed,
could estimate multiple clinically relevant physiological variables like the PSP amplitude and rate constant
\iltodo{rephrase}.
When matured, such a system could have advantages that would set it apart from current methods,
which rely on algorithmic evaluation of extracted EEG features.
In practice, anaesthetic agents like propofol are often administered in combination with analgesic (pain-reducing)
agents (e.g., remifentanil).
This substantially lowers the concentration requirements for the anaesthetic agent to achieve the same level of
sedative effect,
while also helping to increase the quality of anaesthesia, stabilizing the blood pressure and improving recovery
\citationneeded.
\todo{note that current methods might not be able to correctly work with or disentangle the actions and interactions
of multiple drugs }
Being able to model and estimate physiological variables during general anaesthesia
could be a valuable tool for clinical practice.