
\section{Implementation Details}\label{sec:pyrates-framework}

The PyRates Framework is a Python software framework written by Richard Gast and Daniel Rose at
the Max-Plank-Institute in Leipzig.
It can simulate a wide range of graph-representable neural models,
while setting a focus on rate-based population models~\cite{gast_pyratespython_2019}.
It wraps computational backends like Numpy and Tensorflow and offers predefined nodes and
edges (components that model units like cells or cell populations and the
connections between them with mathematical equations) to be used,
replaced or extended with custom equations.
Furthermore, it provides two simple ways to define these components and the derived network configurations:
Either by YAML-File or within Python code.
These configurations are then compiled into optimized executable code with respect to
the chosen backend before being executed.
It comes with pre-configured model-definitions for some of the most frequently used models, e.g., the basic
Jansen-Rit Circuit~\cite{jansen_electroencephalogram_1995} and the
Montbrio-Model~\cite{montbrio_macroscopic_2015}, as well as some variations thereof.
It's ease of use, the fact that it could easily reproduce the characteristics of
the basic Jansen-Rit model out of the box,
and the open-source character made it a sensible choice for this thesis.
%\subsection{Network Representations}\label{subsec:network-representations}
%
%\subsubsection{YAML Representation}
%
%
%\begin{figure}[H]
%	\inputminted[frame=lines, linenos, fontsize=\footnotesize, baselinestretch=1.2, bgcolor=LightGray, tabsize=4]
%	{yaml}{Chapters/Chapter_02_Theoretical_Concepts/code/yaml_synapse.yaml}
%
%	\caption{Example YAML Synapse}\label{fig:yaml_synapse}
%\end{figure}
%
%\begin{figure}[H]
%	\inputminted[frame=lines, linenos, fontsize=\footnotesize, baselinestretch=1.2, bgcolor=LightGray, tabsize=4]
%	{yaml}{Chapters/Chapter_02_Theoretical_Concepts/code/yaml_circuit.yaml}
%
%	\caption{Example YAML Circuit}\label{fig:yaml_circuit}
%\end{figure}
%\subsubsection{Python Representation}
%

\subsection{Implementation of the Basic Jansen-Rit Model}\label{subsec:implementation-of-the-jansen-rit-model}
PyRates works with population models by compositing multiple basic operators
(see Fig.~\ref{fig:pyrates_ops_jr} for the definition of all necessary operators)
like the PSP-Block and Sigmoid-Block into nodes.
These nodes represent populations that can then be connected via edges (synapses).
For example, one might combine two PSP-Blocks (for excitatory and inhibitory input, respectively)
with a Sigmoid Block to create a PC-Node.
This node can then receive rate-input to each of its PSP-Blocks and produce rate-output from its Sigmoid-Block.
The EIN- and IIN- nodes are functionally identical and just combine an excitatory PSP-Block with a Sigmoid Block.
By connecting these Blocks and adding random input to the excitatory PSP-Block of the PC-Node,
the simple Jansen-Rit Circuit is already complete.
Eq.~\ref{eq:pyrates_ode_jr} shows the differential equations that are defined by the implementation,
while Fig.~\ref{fig:pyratesJRBlock} shows the corresponding Block Diagram.
The results of Fig.~\ref{fig:jr_c_sweep},
which perfectly replicate~\cite[Fig. 3]{jansen_electroencephalogram_1995},
validated the implementation of the model and completed \textbf{Goal 1}.

\begin{figure}[H]
    \begin{tikzpicture}[
        pc/.style={draw=cyan!80, fill=cyan!5},
        ein/.style={draw=green!80, fill=green!5},
        iin/.style={draw=red!80, fill=red!5},
        pcLabel/.style={font=\small,text=cyan!80},
        einLabel/.style={font=\small,text=green!80},
        iinLabel/.style={font=\small,text=red!80},
        rectNode/.style={draw=black!80, thick},
        roundNode/.style={circle, draw=black!80, thick},
        ]

\pgfdeclarelayer{bg}
\pgfsetlayers{bg,main}
        
 % Nodes
\node[rectNode] (SigmPC) [] {$Sigm$};
\node[rectNode] (SigmEIN) [above left=2cm and 1cm of SigmPC.center]{$Sigm$};
\node[rectNode] (SigmIIN) [below left=2cm and 1cm of SigmPC.center]{$Sigm$};
\node[rectNode] (PSPPC) [right=1.9cm of SigmEIN.east, fill=white]{$h_e(t)$};
\node[rectNode] (PSPPCI) [right=1.9cm of SigmIIN.east, fill=white]{$h_e(t)$};
\node[rectNode] (PSPEIN) [above left= 0.5cm and 2cm of SigmPC.west, fill=white]{$h_e(t)$};
\node[rectNode] (PSPIIN) [below left= 0.5cm and 2cm of SigmPC.west, fill=white]{$h_i(t)$};
\node[rectNode, rounded corners=3mm] (ext) [left=2cm of PSPEIN.west, label={[]:Ext.}]{$p(t)$};
\node (inpIPSP) [left=0.8cm of PSPIIN.west]{};
\node[roundNode] (c1) [above right=1.2cm and 2.5cm of SigmPC.east]{$C_1$};
\node[roundNode] (c2) [left=1cm of SigmEIN.west]{$C_2$};
\node[roundNode] (c3) [below right=1.2cm and 2.5cm of SigmPC.east]{$C_3$};
\node[roundNode] (c4) [left=1cm of SigmIIN.west]{$C_4$};

% add PC
\node[roundNode] (addPC) [left=0.8cm of SigmPC.west]{};
\draw[-, black!80, thick] (addPC.north west) -- (addPC.south east);
\draw[-, black!80, thick] (addPC.north east) -- (addPC.south west);
% add Excitatory
\node[roundNode] (addExc) [left=0.8cm of PSPEIN.west]{};
\draw[-, black!80, thick] (addExc.north west) -- (addExc.south east);
\draw[-, black!80, thick] (addExc.north east) -- (addExc.south west);

% add PC -> Sigm PC -> PSP PC
\draw[-{Stealth[scale=1.5]}] (addPC.east) -- (SigmPC.west)node[coordinate, pos=0.5](measurepoint){};
\draw[-{Stealth[scale=1.5]}] (SigmPC.east) -| (c1.south);
\draw[-{Stealth[scale=1.5]}] (SigmPC.east) -| (c3.north);


% y0 -> C1 -> Sigm EIN
\draw[-{Stealth[scale=1.5]}] (c1.north) |- (PSPPC.east);
\draw[-{Stealth[scale=1.5]}] (PSPPC.west) -- (SigmEIN.east) node[pos=0.4, above=0.1cm, fill=pyratesgreen!15, draw=pyratesgreen, text=pyratesgreen]{\small$PSP_{EIN}$};

% y0 -> C3 -> Sigm IIN
\draw[-{Stealth[scale=1.5]}] (c3.south) |- (PSPPCI.east);
\draw[-{Stealth[scale=1.5]}] (PSPPCI.west) -- (SigmIIN.east) node[pos=0.4, below=0.1cm, draw=pyratesdarkred, fill=pyratesdarkred!15, text=pyratesdarkred]{\small$PSP_{IIN}$};


% Sigm EIN -> c2 -> add EXC
\draw[-{Stealth[scale=1.5]}] (SigmEIN.west) -- (c2.east);
\draw[-{Stealth[scale=1.5]}] (c2.west) -| (addExc.north) node[pos=0.9, right]{\small$+$};
% external -> add EXC
\draw[-{Stealth[scale=1.5]}] (ext.east) -- (addExc.west) node[pos=0.9, above]{\small$+$};

% add EXC -> PSP EIN
\draw[-{Stealth[scale=1.5]}] (addExc.east) -- (PSPEIN.west);
% PSP EIN -> add PC
\draw[-{Stealth[scale=1.5]}, fill=none] (PSPEIN.east) -| (addPC.north) node[pos=1, left]{\small$+$} node[pos=0.4, above=0.1cm, draw=pyratespurple, fill=pyratespurple!15, text=pyratespurple]{\small$PSP_{PC_E}$};

% Sigm IIN -> C4 -> PSP IIN
\draw[-{Stealth[scale=1.5]}] (SigmIIN.west) -- (c4.east);
\draw[-] (c4.west) -| (inpIPSP.center);
\draw[-{Stealth[scale=1.5]}] (inpIPSP.center) -- (PSPIIN.west);
% PSP IIN -> add PC
\draw[-{Stealth[scale=1.5]}, fill=none] (PSPIIN.east) -| (addPC.south) node[pos=1, left]{\small$-$} node[pos=0.4, below=0.1cm, draw=pyratesorange, fill=pyratesorange!10, text=pyratesorange]{\small$PSP_{PC_I}$};

% electrode
\draw (measurepoint.north) -- (-0.9,1)node[coordinate,pos=0.9](a){} -- (-0.6,0.9)node[coordinate, pos=0.5](b){} -- cycle;
\node (signal)[above right=0.05cm and 2.0cm of a]{\small$PSP_{PC}$};
\draw (b.center) |- (signal.west);


\begin{scope}[shift={(PSPPC.south west)}]
      \begin{axis}[yscale=0.03, xscale=0.16,
            axis x line=none,
            axis y line=none,
            domain=0:140,
            samples=1001,
            xticklabels=\empty,
          ]
          \addplot [green!80] {0.325*x*e^(-0.1*x)};
        \end{axis}
\end{scope}

\begin{scope}[shift={(PSPPCI.south west)}]
      \begin{axis}[yscale=0.03, xscale=0.16,
            axis x line=none,
            axis y line=none,
            domain=0:140,
            samples=1001,
            xticklabels=\empty,
          ]
          \addplot [green!80] {0.325*x*e^(-0.1*x)};
        \end{axis}
\end{scope}

\begin{scope}[shift={(PSPEIN.south west)}]
      \begin{axis}[yscale=0.03, xscale=0.16,
            axis x line=none,
            axis y line=none,
            domain=0:140,
            samples=1001,
            xticklabels=\empty,
          ]
          \addplot [green!80] {0.325*x*e^(-0.1*x)};
        \end{axis}
\end{scope}

\begin{scope}[shift={(PSPIIN.south west)}]
      \begin{axis}[yscale=0.12, xscale=0.16,
            axis x line=none,
            axis y line=none,
            domain=0:140,
            samples=1001,
            xticklabels=\empty,
          ]
          \addplot [red!80] {1.1*x*e^(-0.05*x)};
        \end{axis}
\end{scope}

\begin{pgfonlayer}{bg}
        
    \filldraw [fill=cyan!2,draw=cyan!40]
        ($ (PSPEIN.center) + (-0.4,0.8) $)
        rectangle ($ (PSPIIN.center) + (3.8,-0.8) $);
    \node [below=0.2cm of SigmPC, text=cyan]{PC};
    
    \filldraw [fill=green!2,draw=green!40]
        ($ (SigmEIN.north) + (-0.7,0.4) $)
        rectangle ($ (PSPPC.south) + (0.4,-0.2) $);
    \node [above=1.4cm of SigmPC, text=green]{EIN};
    
    
    \filldraw [fill=red!2,draw=red!40]
        ($ (SigmIIN.north) +  (-0.7,0.2) $)
        rectangle ($ (PSPPCI.south) + (0.4,-0.4) $);
    \node [below=1.4cm of SigmPC, text=red]{IIN};
\end{pgfonlayer}

\end{tikzpicture}
    \caption{\textbf{Jansen-Rit Block Diagram as implemented in PyRates:} Each population can be
    clearly identified by one or more afferent PSP-Blocks and a
    single Sigmoid that calculates the populations output.
    This approach is more modular and simplifies conceptual understanding while staying mathematically equivalent.
    However, due to the explicit fourth PSP-Block it gives up the performance boost.
    }
    \label{fig:pyratesJRBlock}
\end{figure}

\begin{equation}
	\begin{aligned}
		\frac{d}{dt}PSP_{EIN} &= PSP_{t_{EIN}} \\
		\frac{d}{dt}PSP_{t_{EIN}} &= \fcolorbox{pyratesgreen!80}{pyratesgreen!15}{$ \color{pyratesgreen} \frac{H_e}{\tau_e}\cdot C_1 Sigm[PSP_{PC}]  -\frac{2}{\tau_e} \cdot PSP_{t_{EIN}} - \left(\frac{1}{\tau_e}\right)^2 \cdot PSP_{EIN} $}\\
		\frac{d}{dt}PSP_{IIN} &= PSP_{t_{IIN}} \\
		\frac{d}{dt}PSP_{t_{IIN}} &= \fcolorbox{pyratesdarkred!80}{pyratesdarkred!15}{$ \color{pyratesdarkred}\frac{H_e}{\tau_e} \cdot C_3 Sigm[PSP_{PC}]  -\frac{2}{\tau_e} \cdot PSP_{t_{IIN}} - \left(\frac{1}{\tau_e}\right)^2 \cdot PSP_{IIN} $}\\
		\frac{d}{dt}PSP_{PC_E} &= PSP_{t_{PC_E}} \\
		\frac{d}{dt}PSP_{t_{PC_E}} &= \fcolorbox{pyratespurple!80}{pyratespurple!15}{$ \color{pyratespurple}\frac{H_e}{\tau_e} \cdot (p(t) + C_2 Sigm[PSP_{EIN}])  -\frac{2}{\tau_e} \cdot PSP_{t_{PC_E}} - \left(\frac{1}{\tau_e}\right)^2 \cdot PSP_{PC_E} $}\\
		\frac{d}{dt}PSP_{PC_I} &= PSP_{t_{PC_I}} \\
		\frac{d}{dt}PSP_{t_{PC_I}} &= \fcolorbox{pyratesorange!80}{pyratesorange!10}{$ \color{pyratesorange}\frac{H_i}{\tau_i} \cdot C_4 Sigm[PSP_{IIN}])  -\frac{2}{\tau_i} \cdot PSP_{t_{PC_I}} - \left(\frac{1}{\tau_i}\right)^2 \cdot PSP_{PC_I} $}\\
		PSP_{PC} &= PSP_{PC_E} - PSP_{PC_I}
	\end{aligned}\label{eq:pyrates_ode_jr}
\end{equation}

\begin{figure}[H]

        \inputminted[mathescape, frame=lines, linenos, fontsize=\footnotesize, baselinestretch=1.2,
            bgcolor=LightGray, tabsize=4]
        {python3}{Chapters/Chapter_02_Theoretical_Concepts/code/python_example.py}

	\caption{\textbf{Implementation of all relevant operators.}\\
    Fig.~\ref{fig:pyratesJRBlock} contains three instances of \texttt{rpo\_e},
        a single instance of \texttt{rpo\_i} and three instances of \texttt{pro}.}\label{fig:pyrates_ops_jr}
\end{figure}

\subsection{Implementation of Subpopulations}\label{subsec:implementation-of-subpopulations}
Once the coupled differential equations (Eq.~\ref{eq:davidfriston_subpops}) are properly defined,
the subpopulation concept proposed by David and Friston can be seamlessly implemented in PyRates.
Fig.~\ref{fig:subpop_pyrates} shows the necessary adjustments to a generic \texttt{rpo}-operator template
with two subpopulations,
which can be used with values for $h_e$ or $h_i$ respectively.
The final block diagram for the DF model in pictured in Fig.~\ref{fig:pyratesDFBlock}.
To ensure correct implementation,
multiple results from the original work of David \& Friston ~\cite[Fig. 4, Fig. 5, Fig. 6]{david_neural_2003}
were successfully reproduced and concluded \textbf{Goal 2}.
As errors in implementation caused crucial (but without direct comparison hardly detectable) derivations in model behavior
multiple times, this proved to be an invaluable feedback step.

\begin{figure}[H]

        \inputminted[mathescape, frame=lines, linenos, fontsize=\footnotesize, baselinestretch=1.2,
            bgcolor=LightGray, tabsize=4]
        {python3}{Chapters/Chapter_02_Theoretical_Concepts/code/psp_subpop.py}

	\caption{PSP Block with two subpopulations in PyRates}\label{fig:subpop_pyrates}
\end{figure}

\begin{figure}[H]
    \begin{tikzpicture}[
        pc/.style={draw=cyan!80, fill=cyan!5},
        ein/.style={draw=green!80, fill=green!5},
        iin/.style={draw=red!80, fill=red!5},
        pcLabel/.style={font=\small,text=cyan!80},
        einLabel/.style={font=\small,text=green!80},
        iinLabel/.style={font=\small,text=red!80},
        rectNode/.style={draw=black!80, thick},
        roundNode/.style={circle, inner sep=2pt, draw=black!80, thick},
        ]

\pgfdeclarelayer{bg}
\pgfsetlayers{bg,main}
        
 % Nodes
\node[rectNode] (SigmPC) [] {$Sigm$};
\node[rectNode] (SigmEIN) [above left=2cm and 1cm of SigmPC.center]{$Sigm$};
\node[rectNode] (SigmIIN) [below left=2cm and 1cm of SigmPC.center]{$Sigm$};


%% SUBPOPULATIONS FOR EIN
\node[circle, inner sep=1pt, draw, scale=0.3] (out_PSP_PC_e) [right=2.2cm of SigmEIN.east]{};
\draw[-, black!80] (out_PSP_PC_e.west) -- (out_PSP_PC_e.east);
\draw[-, black!80] (out_PSP_PC_e.north) -- (out_PSP_PC_e.south);
\node[circle, draw, thin, inner sep=1pt, above right=0.1cm and 0.15cm of out_PSP_PC_e.center] (PSPPCw1) {\tiny$w_1$};
\node[circle, draw, thin, inner sep=1pt, below right=0.1cm and 0.15cm of out_PSP_PC_e.center] (PSPPCw2) {\tiny$w_2$};
\node[rectNode] (PSPPC1) [right=0.2cm of PSPPCw1.east, thin, inner sep=1pt, fill=white]{\tiny$h_{e_1}(t)$};
\node[rectNode] (PSPPC2) [right=0.2cm of PSPPCw2.east, thin, inner sep=1pt, fill=white]{\tiny$h_{e_2}(t)$};
\node (in_PSP_PC_e) [right=1.38cm of out_PSP_PC_e.center]{};
\node[draw, thick, inner xsep=0.05cm, inner ysep=0.1cm,
      fit=(PSPPC1) (PSPPC2) (PSPPCw1) (PSPPCw2) (in_PSP_PC_e) (out_PSP_PC_e)] (PSPPC) {};


%% SUBPOPULATIONS FOR IIN
\node[circle, inner sep=1pt, draw, scale=0.3] (out_PSP_PC_i) [right=2.2cm of SigmIIN.east]{};
\draw[-, black!80] (out_PSP_PC_i.west) -- (out_PSP_PC_i.east);
\draw[-, black!80] (out_PSP_PC_i.north) -- (out_PSP_PC_i.south);
\node[circle, draw, thin, inner sep=1pt, above right=0.1cm and 0.15cm of out_PSP_PC_i.center] (PSPPCIw1) {\tiny$w_1$};
\node[circle, draw, thin, inner sep=1pt, below right=0.1cm and 0.15cm of out_PSP_PC_i.center] (PSPPCIw2) {\tiny$w_2$};
\node[rectNode] (PSPPCI1) [right=0.2cm of PSPPCIw1.east, thin, inner sep=1pt, fill=white]{\tiny$h_{e_1}(t)$};
\node[rectNode] (PSPPCI2) [right=0.2cm of PSPPCIw2.east, thin, inner sep=1pt, fill=white]{\tiny$h_{e_2}(t)$};
\node (in_PSP_PC_i) [right=1.38cm of out_PSP_PC_i.center]{};
\node[draw, thick, inner xsep=0.05cm, inner ysep=0.1cm,
      fit=(PSPPCI1) (PSPPCI2) (PSPPCIw1) (PSPPCIw2) (in_PSP_PC_i) (out_PSP_PC_i)] (PSPPCI) {};

%% SUBPOPULATIONS FOR PSP_EIN
\node (psp_ein_anchor) [above left= 0.7cm and 3.1cm of SigmPC.west]{};
\node (in_PSP_EIN) [left=0.1cm of psp_ein_anchor)]{};
\node[circle, inner sep=1pt, draw, scale=0.3] (out_PSP_EIN) [right=1.38cm of in_PSP_EIN)]{};
\draw[-, black!80] (out_PSP_EIN.west) -- (out_PSP_EIN.east);
\draw[-, black!80] (out_PSP_EIN.north) -- (out_PSP_EIN.south);
\node[circle, draw, thin, inner sep=1pt, above left=0.1cm and 0.15cm of out_PSP_EIN.center] (PSP_EINw1) {\tiny$w_1$};
\node[circle, draw, thin, inner sep=1pt, below left=0.1cm and 0.15cm of out_PSP_EIN.center] (PSP_EINw2) {\tiny$w_2$};
\node[rectNode] (PSP_EIN1) [left=0.2cm of PSP_EINw1.west, thin, inner sep=1pt, fill=white]{\tiny$h_{e_1}(t)$};
\node[rectNode] (PSP_EIN2) [left=0.2cm of PSP_EINw2.west, thin, inner sep=1pt, fill=white]{\tiny$h_{e_2}(t)$};

\node[draw, thick, inner xsep=0.05cm, inner ysep=0.1cm,
      fit=(PSP_EIN1) (PSP_EIN2) (PSP_EINw1) (PSP_EINw2) (in_PSP_EIN) (out_PSP_EIN)] (PSPEIN) {};

%% SUBPOPULATIONS FOR PSP_IIN
\node (psp_iin_anchor) [below left= 0.7cm and 3.1cm of SigmPC.west]{};
\node (in_PSP_IIN) [left=0.1cm of psp_iin_anchor)]{};
\node[circle, inner sep=1pt, draw, scale=0.3] (out_PSP_IIN) [right=1.38cm of in_PSP_IIN)]{};
\draw[-, black!80] (out_PSP_IIN.west) -- (out_PSP_IIN.east);
\draw[-, black!80] (out_PSP_IIN.north) -- (out_PSP_IIN.south);
\node[circle, draw, thin, inner sep=1pt, above left=0.1cm and 0.15cm of out_PSP_IIN.center] (PSP_IINw1) {\tiny$w_1$};
\node[circle, draw, thin, inner sep=1pt, below left=0.1cm and 0.15cm of out_PSP_IIN.center] (PSP_IINw2) {\tiny$w_2$};
\node[rectNode] (PSP_IIN1) [left=0.2cm of PSP_IINw1.west, thin, inner sep=1pt, fill=white]{\tiny$h_{i_1}(t)$};
\node[rectNode] (PSP_IIN2) [left=0.2cm of PSP_IINw2.west, thin, inner sep=1pt, fill=white]{\tiny$h_{i_2}(t)$};

\node[draw, thick, inner xsep=0.05cm, inner ysep=0.1cm,
      fit=(PSP_IIN1) (PSP_IIN2) (PSP_IINw1) (PSP_IINw2) (in_PSP_IIN) (out_PSP_IIN)] (PSPIIN) {};



\node[rectNode, rounded corners=3mm] (ext) [left=2cm of PSPEIN.west, label={[]:Ext.}]{$p(t)$};
\node (inpIPSP) [left=0.8cm of PSPIIN.west]{};
\node[roundNode] (c1) [above right=0.8cm and 2.5cm of SigmPC.east]{$C_1$};
\node[roundNode] (c2) [left=1cm of SigmEIN.west]{$C_2$};
\node[roundNode] (c3) [below right=0.8cm and 2.5cm of SigmPC.east]{$C_3$};
\node[roundNode] (c4) [left=1cm of SigmIIN.west]{$C_4$};

% add PC
\node[roundNode] (addPC) [left=0.8cm of SigmPC.west]{};
%\draw[-, black!80, thick] (addPC.north west) -- (addPC.south east);
%\draw[-, black!80, thick] (addPC.north east) -- (addPC.south west);
% add Excitatory
\node[roundNode] (addExc) [left=0.8cm of PSPEIN.west]{};
\draw[-, black!80, thick] (addExc.west) -- (addExc.east);
\draw[-, black!80, thick] (addExc.north) -- (addExc.south);

% add PC -> Sigm PC -> PSP PC
\draw[-{Stealth[scale=1.5]}] (addPC.east) -- (SigmPC.west)node[coordinate, pos=0.5](measurepoint){};
\draw[-{Stealth[scale=1.5]}] (SigmPC.east) -| (c1.south);
\draw[-{Stealth[scale=1.5]}] (SigmPC.east) -| (c3.north);


% y0 -> C1 -> Sigm EIN
\draw[-] (c1.north) |- (in_PSP_PC_e.center);
\draw[-{Stealth[scale=0.5]}] (in_PSP_PC_e.center) |- (PSPPC1.east);
\draw[-{Stealth[scale=0.5]}] (in_PSP_PC_e.center) |- (PSPPC2.east);
\draw[-{Stealth[scale=0.5]}] (PSPPC1.west) -- (PSPPCw1.east);
\draw[-{Stealth[scale=0.5]}] (PSPPC2.west) -- (PSPPCw2.east);
\draw[-{Stealth[scale=0.5]}] (PSPPCw1.west) -| (out_PSP_PC_e.north);
\draw[-{Stealth[scale=0.5]}] (PSPPCw2.west) -| (out_PSP_PC_e.south);
\draw[-{Stealth[scale=1.5]}] (out_PSP_PC_e.west) -- (SigmEIN.east) node[pos=0.4, above=0.1cm, fill=pyratesgreen!15,
draw=pyratesgreen, text=pyratesgreen]{\tiny$PSP_{EIN}$};

% y0 -> C3 -> Sigm IIN
\draw[-] (c3.south) |- (in_PSP_PC_i.center);
\draw[-{Stealth[scale=0.5]}] (in_PSP_PC_i.center) |- (PSPPCI1.east);
\draw[-{Stealth[scale=0.5]}] (in_PSP_PC_i.center) |- (PSPPCI2.east);
\draw[-{Stealth[scale=0.5]}] (PSPPCI1.west) -- (PSPPCIw1.east);
\draw[-{Stealth[scale=0.5]}] (PSPPCI2.west) -- (PSPPCIw2.east);
\draw[-{Stealth[scale=0.5]}] (PSPPCIw1.west) -| (out_PSP_PC_i.north);
\draw[-{Stealth[scale=0.5]}] (PSPPCIw2.west) -| (out_PSP_PC_i.south);
\draw[-{Stealth[scale=1.5]}] (out_PSP_PC_i.west) -- (SigmIIN.east) node[pos=0.4, below=0.1cm, draw=pyratesdarkred,
fill=pyratesdarkred!15, text=pyratesdarkred]{\tiny$PSP_{IIN}$};


% Sigm EIN -> c2 -> add EXC
\draw[-{Stealth[scale=1.5]}] (SigmEIN.west) -- (c2.east);
\draw[-{Stealth[scale=1.5]}] (c2.west) -| (addExc.north) node[pos=0.9, right]{\small$+$};
% external -> add EXC
\draw[-{Stealth[scale=1.5]}] (ext.east) -- (addExc.west) node[pos=0.9, above]{\small$+$};

% add EXC -> PSP EIN
\draw[-] (addExc.east) -- (in_PSP_EIN.center);
% inside PSP EIN
\draw[-{Stealth[scale=0.5]}] (in_PSP_EIN.center) |- (PSP_EIN1.west);
\draw[-{Stealth[scale=0.5]}] (in_PSP_EIN.center) |- (PSP_EIN2.west);
\draw[-{Stealth[scale=0.5]}] (PSP_EIN1.east) -- (PSP_EINw1.west);
\draw[-{Stealth[scale=0.5]}] (PSP_EIN2.east) -- (PSP_EINw2.west);
\draw[-{Stealth[scale=0.5]}] (PSP_EINw1.east) -| (out_PSP_EIN.north);
\draw[-{Stealth[scale=0.5]}] (PSP_EINw2.east) -| (out_PSP_EIN.south);
% PSP EIN -> add PC
\draw[-{Stealth[scale=1.5]}, fill=none] (out_PSP_EIN.east) -| (addPC.north) node[pos=1, left]{\small$+$} node[pos=0.4, above=0.1cm, draw=pyratespurple, fill=pyratespurple!15, text=pyratespurple]{\tiny$PSP_{PC_E}$};

% Sigm IIN -> C4 -> PSP IIN
\draw[-{Stealth[scale=1.5]}] (SigmIIN.west) -- (c4.east);
\draw[-] (c4.west) -| (inpIPSP.center);
\draw[-] (inpIPSP.center) -- (in_PSP_IIN.center);
% inside PSP IIN
\draw[-{Stealth[scale=0.5]}] (in_PSP_IIN.center) |- (PSP_IIN1.west);
\draw[-{Stealth[scale=0.5]}] (in_PSP_IIN.center) |- (PSP_IIN2.west);
\draw[-{Stealth[scale=0.5]}] (PSP_IIN1.east) -- (PSP_IINw1.west);
\draw[-{Stealth[scale=0.5]}] (PSP_IIN2.east) -- (PSP_IINw2.west);
\draw[-{Stealth[scale=0.5]}] (PSP_IINw1.east) -| (out_PSP_IIN.north);
\draw[-{Stealth[scale=0.5]}] (PSP_IINw2.east) -| (out_PSP_IIN.south);
% PSP IIN -> add PC
\draw[-{Stealth[scale=1.5]}, fill=none] (out_PSP_IIN.east) -| (addPC.south) node[pos=1, left]{\small$-$} node[pos=0.4, below=0.1cm, draw=pyratesorange, fill=pyratesorange!10, text=pyratesorange]{\tiny$PSP_{PC_I}$};

% electrode
\draw (measurepoint.north) -- (-0.9,1)node[coordinate,pos=0.9](a){} -- (-0.6,0.9)node[coordinate, pos=0.5](b){} -- cycle;
\node (signal)[above right=0.05cm and 2.0cm of a]{\tiny$PSP_{PC}$};
\draw (b.center) |- (signal.west);


%\begin{scope}[shift={(PSPPC1.south west)}]
%      \begin{axis}[yscale=0.03, xscale=0.1,
%            axis x line=none,
%            axis y line=none,
%            domain=0:140,
%            samples=1001,
%            xticklabels=\empty,
%          ]
%          \addplot [green!80] {0.325*x*e^(-0.1*x)};
%        \end{axis}
%\end{scope}
%
%\begin{scope}[shift={(PSPPCI.south west)}]
%      \begin{axis}[yscale=0.03, xscale=0.16,
%            axis x line=none,
%            axis y line=none,
%            domain=0:1400,
%            samples=1001,
%            xticklabels=\empty,
%          ]
%          \addplot [green!80] {0.325*x*e^(-0.1*x)};
%        \end{axis}
%\end{scope}
%
%\begin{scope}[shift={(PSPEIN.south west)}]
%      \begin{axis}[yscale=0.03, xscale=0.16,
%            axis x line=none,
%            axis y line=none,
%            domain=0:140,
%            samples=1001,
%            xticklabels=\empty,
%          ]
%          \addplot [green!80] {0.325*x*e^(-0.1*x)};
%        \end{axis}
%\end{scope}
%
%\begin{scope}[shift={(PSPIIN.south west)}]
%      \begin{axis}[yscale=0.12, xscale=0.16,
%            axis x line=none,
%            axis y line=none,
%            domain=0:140,
%            samples=1001,
%            xticklabels=\empty,
%          ]
%          \addplot [red!80] {1.1*x*e^(-0.05*x)};
%        \end{axis}
%\end{scope}

\begin{pgfonlayer}{bg}

    \node[draw=cyan!40, fill=cyan!2, inner xsep=0.15cm, inner ysep=0.15cm,
      fit=(PSPEIN) (SigmPC) (PSPIIN) ] {};
%    \filldraw [fill=cyan!2,draw=cyan!40]
%        ($ (PSPEIN.center) + (-0.4,0.8) $)
%        rectangle ($ (PSPIIN.center) + (3.8,-0.8) $);
    \node [below=0.2cm of SigmPC, text=cyan]{PC};

    \node[draw=green!40, fill=green!2, inner xsep=0.15cm, inner ysep=0.15cm,
      fit=(PSPPC) (SigmEIN) ] {};
%    \filldraw [fill=green!2,draw=green!40]
%        ($ (SigmEIN.north) + (-0.7,0.4) $)
%        rectangle ($ (PSPPC.south) + (0.4,-0.2) $);
    \node [above=1.4cm of SigmPC, text=green]{EIN};
    
    \node[draw=red!40, fill=red!2, inner xsep=0.15cm, inner ysep=0.15cm,
      fit=(PSPPCI) (SigmIIN) ] {};
%    \filldraw [fill=red!2,draw=red!40]
%        ($ (SigmIIN.north) +  (-0.7,0.2) $)
%        rectangle ($ (PSPPCI.south) + (0.4,-0.4) $);
    \node [below=1.4cm of SigmPC, text=red]{IIN};
\end{pgfonlayer}

\end{tikzpicture}
    \caption{\textbf{David-Friston Block Diagram as implemented in PyRates:}\\
        Each PSP-Block now consists of a slow and a fast subpopulation.
    }
    \label{fig:pyratesDFBlock}
\end{figure}


\subsection{Implementing the Effects of Propofol}\label{subsec:implementing-the-effects-of-propofol}

As described in Sec.~\ref{sec:general-anaesthesia},
propofol modulates the GABA\textsubscript{A} receptors and prolongs the decay-time of the IPSP.
In our model, the inhibitory response function $h_i$ is a direct correlate of the IPSP.
To simulate the effects of propofol on $h_i$, the time-constant $\tau_i$ is simply increased by a factor
$\lambda$:

\[ h_i(t)=\frac{H_i}{\lambda \cdot \tau_i}te^{-\frac{1}{\lambda \cdot \tau_i}} \]
For the subpopulation model, this applies to $h_{i_0}$ and $h_{i_1}$
Varying $\lambda$ between $1$ and $3$ appears to be a sensible choice for the clinically relevant range,
given that $\SI{10}{\micro\molar}$ doubles the decay-time and roughly corresponds with LOC.
Additionally, Steyn-Ross et al.\ use the same range in their model~\cite{hutt_progress_2011}.
The effect of increasing $\lambda$ for $h_{i_0}$ and $h_{i_1}$ is visualized in Fig.~\ref{fig:PSPInhibLongPlot}.

\begin{figure}[H]
    \centering
    \pgfplotsset{compat = newest}
    \begin{tikzpicture}
        \begin{axis}
            [
            xmin = -1, xmax = 280,
            ymin = 0, ymax = 58.1,
            xlabel = {$t$ ($\SI{}{\milli\second}$)},
            ylabel = {$h$ ($\SI{}{\milli\volt}$)},
            legend pos=north east,
            legend style={nodes={scale=0.8, transform shape}},
            domain = 0:280,
            samples = 200,
            smooth,
            thick,
            ],
            \addplot[red] {(20/22)*x*e^(-(1/22)*x)};\label{plot:psp3}
            \addplot[orange] {(151.72/2.9)*x*e^(-(1/2.9)*x)};\label{plot:psp4}
            \addplot[red, dash dot] {(20/44)*x*e^(-(1/44)*x)};\label{plot:psp5.10}
            \addplot[red, dashed] {(20/66)*x*e^(-(1/66)*x)};\label{plot:psp5.15}

            \addplot[orange, dotted]{(151.72/5.8)*x*e^(-(1/5.8)*x)};\label{plot:psp6.10}
            \addplot[orange, dashed] {(151.72/8.7)*x*e^(-(1/8.7)*x)};\label{plot:psp6.15}
            \coordinate (legend) at (axis description cs:0.97,0.97);
        \end{axis}
        \tiny
        \matrix [
            draw,
            matrix of nodes,
            anchor=north east,
        ] at (legend) {

            & $\tau$ &    $H$  \\
            $h_{i_0}(t)$~\ref{plot:psp3} & 22.0ms &  20.0mV \\
            $h_{i_0}(t)$~\ref{plot:psp5.10} & 44.0ms &  20.0mV \\
            $h_{i_0}(t)$~\ref{plot:psp5.15} & 66.0ms &  20.0mV \\
            $h_{i_1}(t)$~\ref{plot:psp4} &  2.9ms & 151.7mV \\
            $h_{i_1}(t)$~\ref{plot:psp6.10} &  5.8ms & 151.7mV \\
            $h_{i_1}(t)$~\ref{plot:psp6.15} &  8.7ms & 151.7mV \\
        };
    \end{tikzpicture}

    \caption{\textbf{Inhibitory PSP functions with varying $\lambda$:} \\
        The duration of the effect increases while the amplitude stays
        constant, effectively increasing the charge transfer. ($\lambda$ in [1.0 (no drug-effect, solid lines),
        2.0 (dotted), 3.0 (dashed)])
    }
    \label{fig:PSPInhibLongPlot}
\end{figure}
