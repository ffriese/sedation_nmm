%! suppress = EscapeUnderscore

\section{Data Analysis}\label{sec:data-analysis}
The NMM implemented in this thesis is configured to simulate data with a step-size of $\SI{1}{\milli\second}$,
yielding a $\SI{1000}{\hertz}$ signal.
The data generated by NMMs has a clear advantage to real EEG recordings:
It has no missing samples,
was never recorded with insufficient impedance,
and there are no artifacts or excessive noise.
That means there is nothing degrading the signal quality,
which would require post-processing steps to try to mitigate these problems
before further analysis.
Nonetheless, it is important to understand the properties of the simulated data to correctly work with it.

When starting a run, with all its state-variables uninitialized,
the system reacts with high-amplitude oscillations to the "disturbance" of the random input.
Usually, it stabilizes very quickly and then exhibits the expected behaviour.
Thus, the initial section of simulated data
is always discarded before further analysis (stabilization-time may vary greatly with parameterization).
This is not unlike to some physical EEG recording hardware,
were the built-in filters require time to initialize before they produce a usable signal.
In our case, removing a few seconds is enough (see Fig.~\ref{fig:initial_oscilations}).

Having removed any unwanted oscillations from the beginning of the recording,
we can turn to our actual analysis of the data:
For
\todo{decide what to do with this section}
\begin{figure}[H]
    \centering
    \begin{tikzpicture}
        \pgfplotsset{
        %% Axis
            scale only axis,
            width=0.4\linewidth,
            height=4cm,
            every axis/.append style={
                line width=1pt,
                tick style={line width=0.8pt},
                %   grid style={dashed, black!20},
                %  grid=major,
            },
%               %% X-Axis
            xmin=-0.1,
            xmax=5,
        }
        \begin{groupplot} [
                group style={
                    group size=2 by 2,
                    vertical sep=12mm,
                    horizontal sep=12mm,
                },
                yticklabel style={
                    /pgf/number format/fixed,
                    /pgf/number format/precision=2
                },
                legend style={nodes={scale=0.8, transform shape}, thin},
                legend image post style={scale=0},
            ]
            \nextgroupplot[ylabel=$\SI{}{\milli\volt}$, xlabel=$s$]
            \addplot [line width=.5pt,solid, cyan]
            table[x=x,y=y ,col sep=comma]{data/methodology/uncut.csv};
            \legend{\textbf{A)} raw data};

            \nextgroupplot[xmin=2.0, xlabel=$s$]
            \addplot [line width=.5pt,solid, cyan]
            table[x=x,y=y ,col sep=comma]{data/methodology/uncut.csv};
            \legend{\textbf{B)} cut data};
        \end{groupplot}
    \end{tikzpicture}

    \caption{
        \textbf{Processing of simulated data.} \\
        Removing initially unstable signal by cutting off the first $\SI{2}{\second}$ of the
        data (generated by JR Model).
    }
    \label{fig:initial_oscilations}
\end{figure}