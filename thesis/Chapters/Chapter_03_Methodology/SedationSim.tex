\section{Simulating the effects of propofol}\label{sec:simulating-the-effects-of-propofol}


\section{Implementing decay-time modulation}\label{sec:implementing-decay-time-modulation}

To simulate the effects of propofol on the GABA\textsubscript{A}  receptors,
the IPSP (inhibitory response function $h_i$) time-constant $\tau_i$ is increased by a factor
$\lambda$~\cite{hutt_effects_2010}:


\[ h_i(t)=\frac{H_i}{\lambda \cdot \tau_i}te^{-\frac{1}{\lambda \cdot \tau_i}} \]
The effect of increasing $\lambda$ for $h_{i_1}$ and $h_{i_2}$ is visualized in Fig.~\ref{fig:PSPInhibLongPlot}.


\begin{figure}[H]
    \centering
    \pgfplotsset{compat = newest}
    \begin{tikzpicture}
        \begin{axis}
            [
            xmin = -1, xmax = 280,
            ymin = 0, ymax = 58.1,
            xlabel = {$t$ ($ms$)},
            ylabel = {$h$ ($mV$)},
            legend pos=north east,
            legend style={nodes={scale=0.8, transform shape}},
            domain = 0:280,
            samples = 200,
            smooth,
            thick,
            ],
            \addplot[red] {(20/22)*x*e^(-(1/22)*x)};\label{plot:psp3}
            \addplot[orange] {(151.72/2.9)*x*e^(-(1/2.9)*x)};\label{plot:psp4}
            \addplot[red, dash dot] {(20/44)*x*e^(-(1/44)*x)};\label{plot:psp5.10}
            \addplot[red, dashed] {(20/66)*x*e^(-(1/66)*x)};\label{plot:psp5.15}

            \addplot[orange, dotted]{(151.72/5.8)*x*e^(-(1/5.8)*x)};\label{plot:psp6.10}
            \addplot[orange, dashed] {(151.72/8.7)*x*e^(-(1/8.7)*x)};\label{plot:psp6.15}
            \coordinate (legend) at (axis description cs:0.97,0.97);
        \end{axis}
        \tiny
        \matrix [
            draw,
            matrix of nodes,
            anchor=north east,
        ] at (legend) {

            & $\tau$ &    $H$  \\
            $h_{i_1}(t)$~\ref{plot:psp3} & 22.0ms &  20.0mV \\
            $h_{i_1}(t)$~\ref{plot:psp5.10} & 44.0ms &  20.0mV \\
            $h_{i_1}(t)$~\ref{plot:psp5.15} & 66.0ms &  20.0mV \\
            $h_{i_2}(t)$~\ref{plot:psp4} &  2.9ms & 151.7mV \\
            $h_{i_2}(t)$~\ref{plot:psp6.10} &  5.8ms & 151.7mV \\
            $h_{i_2}(t)$~\ref{plot:psp6.15} &  8.7ms & 151.7mV \\
        };
    \end{tikzpicture}

    \caption{\textbf{Inhibitory PSP functions with varying $\lambda$:} \\
        The duration of the effect increases while the amplitude stays
        constant, effectively increasing the charge transfer. ($\lambda$ in [1.0 (no drug-effect, solid lines),
        2.0 (dotted), 3.0 (dashed)])
    }
    \label{fig:PSPInhibLongPlot}
\end{figure}
Varying $\lambda$ between $1$ ($\SI{0}{\micro\molar}$) and $3.0$ ($\sim\SI{30}{\micro\molar}$) appears to be a
sensible choice for the clinically relevant range, given Fig.~\ref{fig:lambda_fit}.
