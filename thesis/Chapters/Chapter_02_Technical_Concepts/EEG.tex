\section{EEG}\label{sec:eeg}

\subsection{Measurement}\label{subsec:measurement}
\qquad \todo{how the EEG is measured technically and which neuronal processes it actually
observes (signal amplification, pyramidal cells, ...)}




\subsection{Advantages/Disadvantages}\label{subsec:advantages/disadvantages}
\qquad \todo{spatial/temporal resolution, invasiveness, ...}
\subsection{States of Consciousness in the EEG Signal}\label{subsec:states-of-consciousness-in-the-eeg-signal}
\qquad \todo{how do the states differ in the signal}




\subsection{Biphasic Effect}

A biphasic effect (an initial increase of an effect, that decreases with higher concentrations) in the EEG can be
observed for many sedatives \cite{kuizenga_quantitative_1998, kuizenga_biphasic_2001}.
For propofol, a temporary steep increase in EEG amplitude in the 2--20 Hz ranges, loosely correlated with the onset of
LOC, as well as ROC can be observed.
% Stage-transition, unstable area between two stable states (consciousness/unconsciousness)




\subsection{Simulation}\label{subsec:simulation}

\subsubsection{Motivation}
 \qquad \qquad \todo{what can we hope to achieve by simulating an EEG signal}
\subsubsection{Approaches}
 \qquad \qquad \todo{which tools are at our disposal (naive frequency mixing,
    population models, simulating individual neurons, ...)}
 \subsubsection{Model Choice}
 \qquad \qquad \todo{argue about models => why did we land on NMMs/population models?}
 
% \begin{itemize}
%	\item What does the EEG measure?
%	\item Which advantages and disadvantages come with this measuring method?
%	\item How do different states of consciousness, especially sedation, appear in the EEG signal?
%	\item How can these states be detected automatically?
%	\begin{itemize}
%		\item How well does that work, where does it fail?
%		\begin{itemize}
%			\item Why is automatic detection hard in subjects with DOC?
%		\end{itemize}
%	\end{itemize}
% \end{itemize}