\chapter{Results}\label{ch:results}
Having accomplished \textbf{Goals 1-3} by implementing the $\lambda$-parameter in our model,
we can set our sights on \textbf{Goal 4} and run a simulation session that mimics GA for each of our two implementations.
As established in Sec.~\ref{subsec:implementing-the-effects-of-propofol},
$\lambda$ may vary between 1 and 3 to approximate the effects of clinically relevant propofol concentrations.
A time-course that closely resembles a real concentration-progression during GA seems intuitively desirable,
but this would complicate a structured analysis of the data,
since induction, maintenance, and emergence from anaesthesia
have a highly asymmetrical concentration development.
While a fully realistic parameter development might be interesting as a next step,
the general properties of our model under the presence of specific
concentrations of a GABA\textsubscript{A}-potentiating agent
are arguably best explored in a systematic fashion, that still follows the general concept of induction and emergence:
A short baseline phase, followed by a linear rise to peak dosage,
a period of maintenance, and a linear decay to the initial condition,
which is again shortly maintained.

%\section{Exploring realistic parameter ranges}\label{sec:simulation-over-the-parameter-space}

Simulating this time-course over the selected parameter space
$ \lambda \in \left[ 1, 3 \right] $ for $\SI{151}{\second}$ (see Fig~\ref{fig:sedation_sim_jr}a),
yields a $\SI{1000}{\hertz}$ signal that includes the expected transients described in Sec.~\ref{sec:data-analysis},
which are consequently removed before continuing.
This circumstance is naturally regarded in the generation of the time-course,
that pads the start of the simulation with an additional second of the baseline state.
The resulting $\SI{150}{\second}$ signal is used to calculate a time-frequency spectrogram
using the python library function \texttt{matplotlib.pyplot.specgram} with a window-size of \texttt{2048}
and overlap of \texttt{1798} (\texttt{window\_size-(0.25*freq\_hz)}).
Supporting data in the form of absolute and relative band-power plots can be found in the appendix
(Fig.~\ref{fig:abs_bands}, Fig.~\ref{fig:rel_bands}).
\newtoggle{drawLocRoc}

\toggletrue{drawLocRoc}
\def\simRunName{JR_SEDATION_150}
\def\simRunTime{151}
\def\locStart{1.1}
\def\locST{10.07}%6.61
\def\locEnd{2.07}
\def\locET{40.38}%18.89
\def\rocStart{2.0}
\def\rocST{112.88}%48.63
\def\rocEnd{1.00}
\def\rocET{144.46}%62.0
\section{Phenomenology of the Basic JR Model}\label{sec:phenomenology-of-the-basic-jr-model}
In the baseline state ($\lambda = 1$), the JR model produces the expected well-defined alpha-activity
    with a sharp frequency peak at $\SI{11}{\hertz}$ (Fig~\ref{fig:sedation_sim_jr}c).
\newremark{Spectrogram Amplitude}{
    At this point, it should be noted
    that even though the spectrogram (Fig~\ref{fig:sedation_sim_jr}c) appears to contain a frequency range
    from roughly 6--18Hz at the beginning of the simulation,
    this is misleading.
    The extremely high-amplitude 12Hz peak in the calculated spectrum has noisy flanks with low amplitudes,
    which are highlighted by the chosen colormap and value range (compare to Fig.~\ref{fig:jr_psd}, Fig.~\ref{fig:jr_spec}
    and the scaled plots Fig.~\ref{fig:jr_psd_rescaled}, Fig.~\ref{fig:jr_spec_rescaled}).
    Their relative contribution to the signal is close to insignificant in this case.
}
    Increasing the simulated propofol levels first leads to the single dominant frequency at $\SI{11}{\hertz} $
    descending towards $\SI{10}{\hertz} $.
    When reaching $\lambda \approx \locStart $,
    there is an onset of heavy oscillations, characterized by dramatically increasing signal amplitude
    with frequencies peaking at around $5, 10, 15$, and $ \SI{20}{\hertz} $,
    but extending up to $\approx \SI{35}{\hertz} $.
    The frequency peaks appear at multiples of the `fundamental` frequency of $\SI{5}{\hertz}$,
    which is a well-known phenomenon in spectral analysis called `harmonics`.
    Further increase of the concentration shifts the base-frequency and its harmonic components' peaks
    slowly towards lower values, preserving their harmonic nature by moving closer to each other.
    This results in added peaks, always closing the gap to $\approx \SI{35}{\hertz} $.
    Additionally, higher values of $\lambda$ appear to linearly decrease the mean signal voltage.
    Another sudden change occurs around $\lambda \approx \locEnd $,
    where the disturbances disappear again,
    with the system having apparently reached a different stable state with very low activity.
    Signal amplitude is drastically reduced, with remaining activity mostly below $\SI{10}{\hertz}$.
    The single dominant frequency has disappeared completely.
    From there on, the mean signal voltage slowly continues to slightly decrease as before.
    However, the frequency distribution appears to have settled.
    After reaching peak dosage at $\lambda = 3.0$, maintaining it for a few seconds has no further effects.
    Subsequent concentration decrease has the expected reverse effects:
    At first only the signal voltage slightly increases,
    then disturbances begin to form again at $\lambda \approx \rocStart$.
    On the return path, effects roughly mirror the induction.
    It takes a few seconds after $\lambda$ returns to its initial value until the initial state is restored.

\begin{figure}[H]
\begin{tikzpicture}
\pgfplotsset{
        %% Axis
            every x tick label/.append style={font=\tiny, yshift=0.5ex},
            every y tick label/.append style={font=\tiny, xshift=0.5ex},
            scale only axis,
            width=0.9\linewidth,
            height=3cm,
            every axis/.append style={
                line width=1pt,
                tick style={line width=0.8pt},
            },
            %% X-Axis
            xmin=1.0, xmax=66.0,
        }
        \begin{groupplot} [
                group style={
                    group size=1 by 3,
                    vertical sep=2mm,
                    xlabels at=edge bottom,
                    xticklabels at=edge bottom,
                },
                yticklabel style={
                    /pgf/number format/fixed,
                    /pgf/number format/precision=2
                },
                legend style={nodes={scale=0.8, transform shape}, thin},
                legend image post style={scale=0},
                xlabel=$t(s)$
            ]
            \nextgroupplot[ylabel style={align=center}, ylabel=\begin{tiny}decay-time factor\end{tiny}\\ $\lambda$,
                           ymin=0.6, ymax=3.1, grid style={dashed,black!20}, grid=major]
            \addplot [name path=factors,line width=.5pt,solid, cyan]
            table[x=x, y=y ,col sep=comma, each nth point=100]{data/full_sedation_sim/\simRunName _factors.csv};

            \legend{\textbf{(a) Propofol concentration}};
        \iftoggle{drawLocRoc}{
                \draw [red, dotted] ([xshift=0.0cm]axis cs:0,\locStart) -- ([yshift=0.0cm]axis cs:\locST,\locStart) ;
                \draw [red, dotted] ([xshift=0.0cm]axis cs:0,\locEnd) -- ([yshift=0.0cm]axis cs:\locET,\locEnd)
            node[near
            end, above,font=\tiny]{LOC $\approx [\locStart,\locEnd]$};
                \draw [red, dotted] ([xshift=0.0cm]axis cs:\locET,0) -- ([yshift=0.0cm]axis cs:\locET,\locEnd);
                \draw [red, dotted] ([xshift=0.0cm]axis cs:\locST,0) -- ([yshift=0.0cm]axis cs:\locST,\locStart);

                \draw [green, dotted] ([xshift=0.0cm]axis cs:66.0,\rocEnd) -- ([yshift=0.0cm]axis cs:\rocET,\rocEnd) ;
                \draw [green, dotted] ([xshift=0.0cm]axis cs:66.0,\rocStart) -- ([yshift=0.0cm]axis cs:\rocST,\rocStart)
            node[near
                       end,above,font=\tiny]{ROC $\approx [\rocStart,\rocEnd]$};
                \draw [green, dotted] ([xshift=0.0cm]axis cs:\rocST,0) -- ([yshift=0.0cm]axis cs:\rocST,\rocStart);
                \draw [green, dotted,line width=.5pt] ([xshift=0.0cm]axis cs:\rocET,0) -- ([yshift=0.0cm]axis cs:\rocET,\rocEnd);
        }

            \nextgroupplot[ylabel=$mV$]
            \addplot [line width=.5pt,solid, cyan]
            table[x=x, y=y ,col sep=comma, each nth point=2]
            {data/full_sedation_sim/\simRunName .csv};

            \legend{\textbf{(b) Signal}};

            \nextgroupplot[ymin=0, ymax=40,
                            ylabel=$Hz$,
                            height=5cm
            ]
              \addplot graphics [includegraphics cmd=\pgfimage,
        xmin=1.0,%xmin=0.899,
        xmax=66.0,% xmax=38.899,
        ymin=0, ymax=40]
            {data/full_sedation_sim/\simRunName -img0.png};
            \legend{\textbf{(c) Spectrogram}};

        \end{groupplot}
     \begin{groupplot} [group style={group size=1 by 3,vertical sep=2mm}]
            \nextgroupplot[axis y line=right, axis line style={-}, ymin=0.6, ymax=3.1, ylabel=$\sim c_e
            (\SI{}{\micro\molar})$,
            yticklabels ={0,5,10,20,30}, ytick={1,1.81,2.164,2.558,2.8},
            axis x line=none]
            \nextgroupplot[axis y line=none, axis x line=none]
            \nextgroupplot[axis y line=right, axis line style={-}, xshift=0.5cm, ymin=0, ymax=40,
                ylabel=$dB$,
                yticklabels ={-110,-85,-40}, ytick={0, 20, 40},
                height=5cm,axis x line=none]
        \end{groupplot}

              \node [above right] at (12.8cm, -8.53cm) {
            \includegraphics[width=0.35cm, height=5cm]
            {data/full_sedation_sim/\simRunName -img1.png}};
\end{tikzpicture}
\caption{\textbf{Simulation of a sedation in the JR Model}
}\label{fig:sedation_sim_jr}
\end{figure}

\toggletrue{drawLocRoc}
\def\simRunName{DF_SEDATION_150}
\def\simRunTime{151}
\def\locStart{1.85}
\def\locST{33.5} % 16.17
\def\locEnd{2.05}
\def\locET{39.75}%18.72
\def\rocStart{1.95}
\def\rocST{114.72}%49.3
\def\rocEnd{1.5}
\def\rocET{128.81}%55.03
\section{Phenomenology of the DF Extension}\label{subsec:phenomenology-of-the-df-extension}
Section~\ref{subsec:the-david-and-friston-model} introduces the David--Friston model and
    the mixture of slow and fast kinetics within a population.
    This results in an initial frequency spectrum
    that lacks the distinct high-amplitude peak and instead has a wider distribution with far lower amplitude.

\newremark{Spectrogram Amplitude}{
    Fig.~\ref{fig:sedation_sim_jr}c and Fig.~\ref{fig:sedation_sim_df}c have the same
    color-map and amplitude-range.
    In contrast to the spectrogram of the JR-Model,
    the low-amplitude activity is significant from the start here,
    as it makes up the whole baseline signal.
}
    The DF model initially produces a dominant frequency range at $10-20 \SI{}{\hertz} $,
    which slowly shifts towards $ 5-10 \SI{}{\hertz} $ (Fig~\ref{fig:sedation_sim_df}c) when $\lambda$ starts to increase.
    The mean signal voltage concurrently decreases,
    while oscillation-amplitude is maintained (Fig~\ref{fig:sedation_sim_df}b).
    The system appears to be in a stable state until $ \lambda \approx \locStart $,
    where the sudden onset of dramatically
    increasing signal amplitude with strongly pronounced activity below $ \SI{25}{\hertz} $
    creates multiple strong harmonic frequency peaks.
    Unlike in the JR model, the peaks do not exceed  $ \SI{30}{\hertz} $ and the fundamental frequency starts off
    at roughly $\SI{3}{\hertz}$.
    Otherwise, the changes to the spectrum are similar to the JR model.
    Further increasing $\lambda$ keeps shifting the fundamental frequency and its harmonics down
    and decreases mean signal voltage.
    The heavy oscillations remain until $\lambda \approx \locEnd $,
    where the dominant frequencies move below $\SI{10}{\hertz}$.
    Continuing, the signal voltage slowly decreases as before,
    however, the frequency distribution appears to have settled.
    Maintaining peak dosage has no further effects.
    Decreasing the simulated propofol levels again first increases the mean signal voltage
    until the stable state dissolves into heavy oscillations around $\lambda \approx \rocStart$.
    The unstable state prevails until $\lambda$ reaches $\approx \rocEnd$.


\begin{figure}[H]
\begin{tikzpicture}
\pgfplotsset{
        %% Axis
            every x tick label/.append style={font=\tiny, yshift=0.5ex},
            every y tick label/.append style={font=\tiny, xshift=0.5ex},
            scale only axis,
            width=0.9\linewidth,
            height=3cm,
            every axis/.append style={
                line width=1pt,
                tick style={line width=0.8pt},
            },
            %% X-Axis
            xmin=1.0, xmax=66.0,
        }
        \begin{groupplot} [
                group style={
                    group size=1 by 3,
                    vertical sep=2mm,
                    xlabels at=edge bottom,
                    xticklabels at=edge bottom,
                },
                yticklabel style={
                    /pgf/number format/fixed,
                    /pgf/number format/precision=2
                },
                legend style={nodes={scale=0.8, transform shape}, thin},
                legend image post style={scale=0},
                xlabel=$t(s)$
            ]
            \nextgroupplot[ylabel style={align=center}, ylabel=\begin{tiny}decay-time factor\end{tiny}\\ $\lambda$,
                           ymin=0.6, ymax=3.1, grid style={dashed,black!20}, grid=major]
            \addplot [name path=factors,line width=.5pt,solid, cyan]
            table[x=x, y=y ,col sep=comma, each nth point=100]{data/full_sedation_sim/\simRunName _factors.csv};

            \legend{\textbf{(a) Propofol concentration}};
        \iftoggle{drawLocRoc}{
                \draw [red, dotted] ([xshift=0.0cm]axis cs:0,\locStart) -- ([yshift=0.0cm]axis cs:\locST,\locStart) ;
                \draw [red, dotted] ([xshift=0.0cm]axis cs:0,\locEnd) -- ([yshift=0.0cm]axis cs:\locET,\locEnd)
            node[near
            end, above,font=\tiny]{LOC $\approx [\locStart,\locEnd]$};
                \draw [red, dotted] ([xshift=0.0cm]axis cs:\locET,0) -- ([yshift=0.0cm]axis cs:\locET,\locEnd);
                \draw [red, dotted] ([xshift=0.0cm]axis cs:\locST,0) -- ([yshift=0.0cm]axis cs:\locST,\locStart);

                \draw [green, dotted] ([xshift=0.0cm]axis cs:66.0,\rocEnd) -- ([yshift=0.0cm]axis cs:\rocET,\rocEnd) ;
                \draw [green, dotted] ([xshift=0.0cm]axis cs:66.0,\rocStart) -- ([yshift=0.0cm]axis cs:\rocST,\rocStart)
            node[near
                       end,above,font=\tiny]{ROC $\approx [\rocStart,\rocEnd]$};
                \draw [green, dotted] ([xshift=0.0cm]axis cs:\rocST,0) -- ([yshift=0.0cm]axis cs:\rocST,\rocStart);
                \draw [green, dotted,line width=.5pt] ([xshift=0.0cm]axis cs:\rocET,0) -- ([yshift=0.0cm]axis cs:\rocET,\rocEnd);
        }

            \nextgroupplot[ylabel=$mV$]
            \addplot [line width=.5pt,solid, cyan]
            table[x=x, y=y ,col sep=comma, each nth point=2]
            {data/full_sedation_sim/\simRunName .csv};

            \legend{\textbf{(b) Signal}};

            \nextgroupplot[ymin=0, ymax=40,
                            ylabel=$Hz$,
                            height=5cm
            ]
              \addplot graphics [includegraphics cmd=\pgfimage,
        xmin=1.0,%xmin=0.899,
        xmax=66.0,% xmax=38.899,
        ymin=0, ymax=40]
            {data/full_sedation_sim/\simRunName -img0.png};
            \legend{\textbf{(c) Spectrogram}};

        \end{groupplot}
     \begin{groupplot} [group style={group size=1 by 3,vertical sep=2mm}]
            \nextgroupplot[axis y line=right, axis line style={-}, ymin=0.6, ymax=3.1, ylabel=$\sim c_e
            (\SI{}{\micro\molar})$,
            yticklabels ={0,5,10,20,30}, ytick={1,1.81,2.164,2.558,2.8},
            axis x line=none]
            \nextgroupplot[axis y line=none, axis x line=none]
            \nextgroupplot[axis y line=right, axis line style={-}, xshift=0.5cm, ymin=0, ymax=40,
                ylabel=$dB$,
                yticklabels ={-110,-85,-40}, ytick={0, 20, 40},
                height=5cm,axis x line=none]
        \end{groupplot}

              \node [above right] at (12.8cm, -8.53cm) {
            \includegraphics[width=0.35cm, height=5cm]
            {data/full_sedation_sim/\simRunName -img1.png}};
\end{tikzpicture}
\caption{\textbf{Simulation of a sedation in the DF Model}
}\label{fig:sedation_sim_df}
\end{figure}

\section{Similarities and Differences}
    Both models share some key features during these simulations.
    Three distinguishable states can be observed in both models;
    The initial stable state is followed by a state that is defined by strong harmonic frequencies and dramatically
    increased signal amplitude.
    From there, the system transitions into a low-power state with low frequencies,
    which seems to be mostly saturated and changes little with further increase of decay-time.
    On the return path the second state is visited again before reentering into the initial state.
    Induction and emergence are asymmetrical, even if only marginally so in the JR model.
    The second state can be regarded as the phase transition between the two stable states.
    The system predicts that the frequency range below $\SI{25}{\hertz}$ receives a temporary amplitude boost during
    these phase transitions, which disappear while the parameter changes continue in the same direction.


There are also a few striking differences between the two simulation sessions:
By design,
the JR model produces exclusively near-sinusoid single-peak alpha-activity in the initial state ($\lambda = 1$),
while the DF model creates a range of alpha- and beta-activity.
The JR model spends significant amounts of the session in the second, high-amplitude state,
while these sections are significantly shorter for the DF model.
Also, in the DF model,
the parameter ranges that cause the two high-amplitude states "LOC" and "ROC" differ decisively.
The "LOC"-transition not only ends at a slightly higher $\lambda$ value ($2.05$) than the start of the
"ROC"-transition ($1.95$).
It also starts far higher ($1.85$) than the "ROC" ends ($1.5$).
Consequently, the "LOC" value-range is far shorter ($\text{length}=0.2$) than the "ROC" range
($\text{length}=0.45$).
In the JR model this effect is present as well, but hardly noticeable,
since the differences are much smaller ($2.07$ vs $2.0$ and $1.1$ vs $1.0$, range-lengths $1.06$ and $1.0$).
Overall, the models differ especially in their frequency composition in the baseline-state
and in their prevalence and symmetry of the transition state.