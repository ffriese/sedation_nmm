\chapter{Results}\label{ch:results}
To generate a signal with the desired effects, it is imperative to design the change of parameter values over the
course of time.

\todo{name goals of simulation}
\todo{explain reasoning behind linear increase (close to realistic sedation, but focus on structural exploration of the
possible values)}
% phasendiagram

%propofol-region/vs nicht propofol region
%yapunov kooefizienten, fractale dimension der attraktoren
%chaotische regionen -> welche dimensionen werden ``gedehnt'' oder ``gestaucht''


% hypothese -> dynamische rahmenbedingen, analyse können bewusstsein nicht erfassen, aber notwendige rahmenbedingungen

% muster in wechsel zwischen rahmenbedingungen erkennen

%

%
%\section{The finally used model}\label{sec:the-finally-used-model}
%
%\begin{figure}[H]
%\begin{tikzpicture}[
        pc/.style={draw=cyan!80, fill=cyan!5},
        ein/.style={draw=green!80, fill=green!5},
        iin/.style={draw=red!80, fill=red!5},
        pcLabel/.style={font=\small,text=cyan!80},
        einLabel/.style={font=\small,text=green!80},
        iinLabel/.style={font=\small,text=red!80},
        rectNode/.style={draw=black!80, thick},
        roundNode/.style={circle, inner sep=2pt, draw=black!80, thick},
        ]

\pgfdeclarelayer{bg}
\pgfsetlayers{bg,main}
        
 % Nodes
\node[rectNode] (SigmPC) [] {$Sigm$};
\node[rectNode] (SigmEIN) [above left=2cm and 1cm of SigmPC.center]{$Sigm$};
\node[rectNode] (SigmIIN) [below left=2cm and 1cm of SigmPC.center]{$Sigm$};


%% SUBPOPULATIONS FOR EIN
\node[circle, inner sep=1pt, draw, scale=0.3] (out_PSP_PC_e) [right=2.2cm of SigmEIN.east]{};
\draw[-, black!80] (out_PSP_PC_e.west) -- (out_PSP_PC_e.east);
\draw[-, black!80] (out_PSP_PC_e.north) -- (out_PSP_PC_e.south);
\node[circle, draw, thin, inner sep=1pt, above right=0.1cm and 0.15cm of out_PSP_PC_e.center] (PSPPCw1) {\tiny$w_1$};
\node[circle, draw, thin, inner sep=1pt, below right=0.1cm and 0.15cm of out_PSP_PC_e.center] (PSPPCw2) {\tiny$w_2$};
\node[rectNode] (PSPPC1) [right=0.2cm of PSPPCw1.east, thin, inner sep=1pt, fill=white]{\tiny$h_{e_1}(t)$};
\node[rectNode] (PSPPC2) [right=0.2cm of PSPPCw2.east, thin, inner sep=1pt, fill=white]{\tiny$h_{e_2}(t)$};
\node (in_PSP_PC_e) [right=1.38cm of out_PSP_PC_e.center]{};
\node[draw, thick, inner xsep=0.05cm, inner ysep=0.1cm,
      fit=(PSPPC1) (PSPPC2) (PSPPCw1) (PSPPCw2) (in_PSP_PC_e) (out_PSP_PC_e)] (PSPPC) {};


%% SUBPOPULATIONS FOR IIN
\node[circle, inner sep=1pt, draw, scale=0.3] (out_PSP_PC_i) [right=2.2cm of SigmIIN.east]{};
\draw[-, black!80] (out_PSP_PC_i.west) -- (out_PSP_PC_i.east);
\draw[-, black!80] (out_PSP_PC_i.north) -- (out_PSP_PC_i.south);
\node[circle, draw, thin, inner sep=1pt, above right=0.1cm and 0.15cm of out_PSP_PC_i.center] (PSPPCIw1) {\tiny$w_1$};
\node[circle, draw, thin, inner sep=1pt, below right=0.1cm and 0.15cm of out_PSP_PC_i.center] (PSPPCIw2) {\tiny$w_2$};
\node[rectNode] (PSPPCI1) [right=0.2cm of PSPPCIw1.east, thin, inner sep=1pt, fill=white]{\tiny$h_{e_1}(t)$};
\node[rectNode] (PSPPCI2) [right=0.2cm of PSPPCIw2.east, thin, inner sep=1pt, fill=white]{\tiny$h_{e_2}(t)$};
\node (in_PSP_PC_i) [right=1.38cm of out_PSP_PC_i.center]{};
\node[draw, thick, inner xsep=0.05cm, inner ysep=0.1cm,
      fit=(PSPPCI1) (PSPPCI2) (PSPPCIw1) (PSPPCIw2) (in_PSP_PC_i) (out_PSP_PC_i)] (PSPPCI) {};

%% SUBPOPULATIONS FOR PSP_EIN
\node (psp_ein_anchor) [above left= 0.7cm and 3.1cm of SigmPC.west]{};
\node (in_PSP_EIN) [left=0.1cm of psp_ein_anchor)]{};
\node[circle, inner sep=1pt, draw, scale=0.3] (out_PSP_EIN) [right=1.38cm of in_PSP_EIN)]{};
\draw[-, black!80] (out_PSP_EIN.west) -- (out_PSP_EIN.east);
\draw[-, black!80] (out_PSP_EIN.north) -- (out_PSP_EIN.south);
\node[circle, draw, thin, inner sep=1pt, above left=0.1cm and 0.15cm of out_PSP_EIN.center] (PSP_EINw1) {\tiny$w_1$};
\node[circle, draw, thin, inner sep=1pt, below left=0.1cm and 0.15cm of out_PSP_EIN.center] (PSP_EINw2) {\tiny$w_2$};
\node[rectNode] (PSP_EIN1) [left=0.2cm of PSP_EINw1.west, thin, inner sep=1pt, fill=white]{\tiny$h_{e_1}(t)$};
\node[rectNode] (PSP_EIN2) [left=0.2cm of PSP_EINw2.west, thin, inner sep=1pt, fill=white]{\tiny$h_{e_2}(t)$};

\node[draw, thick, inner xsep=0.05cm, inner ysep=0.1cm,
      fit=(PSP_EIN1) (PSP_EIN2) (PSP_EINw1) (PSP_EINw2) (in_PSP_EIN) (out_PSP_EIN)] (PSPEIN) {};

%% SUBPOPULATIONS FOR PSP_IIN
\node (psp_iin_anchor) [below left= 0.7cm and 3.1cm of SigmPC.west]{};
\node (in_PSP_IIN) [left=0.1cm of psp_iin_anchor)]{};
\node[circle, inner sep=1pt, draw, scale=0.3] (out_PSP_IIN) [right=1.38cm of in_PSP_IIN)]{};
\draw[-, black!80] (out_PSP_IIN.west) -- (out_PSP_IIN.east);
\draw[-, black!80] (out_PSP_IIN.north) -- (out_PSP_IIN.south);
\node[circle, draw, thin, inner sep=1pt, above left=0.1cm and 0.15cm of out_PSP_IIN.center] (PSP_IINw1) {\tiny$w_1$};
\node[circle, draw, thin, inner sep=1pt, below left=0.1cm and 0.15cm of out_PSP_IIN.center] (PSP_IINw2) {\tiny$w_2$};
\node[rectNode] (PSP_IIN1) [left=0.2cm of PSP_IINw1.west, thin, inner sep=1pt, fill=white]{\tiny$h_{i_1}(t)$};
\node[rectNode] (PSP_IIN2) [left=0.2cm of PSP_IINw2.west, thin, inner sep=1pt, fill=white]{\tiny$h_{i_2}(t)$};

\node[draw, thick, inner xsep=0.05cm, inner ysep=0.1cm,
      fit=(PSP_IIN1) (PSP_IIN2) (PSP_IINw1) (PSP_IINw2) (in_PSP_IIN) (out_PSP_IIN)] (PSPIIN) {};



\node[rectNode, rounded corners=3mm] (ext) [left=2cm of PSPEIN.west, label={[]:Ext.}]{$p(t)$};
\node (inpIPSP) [left=0.8cm of PSPIIN.west]{};
\node[roundNode] (c1) [above right=0.8cm and 2.5cm of SigmPC.east]{$C_1$};
\node[roundNode] (c2) [left=1cm of SigmEIN.west]{$C_2$};
\node[roundNode] (c3) [below right=0.8cm and 2.5cm of SigmPC.east]{$C_3$};
\node[roundNode] (c4) [left=1cm of SigmIIN.west]{$C_4$};

% add PC
\node[roundNode] (addPC) [left=0.8cm of SigmPC.west]{};
%\draw[-, black!80, thick] (addPC.north west) -- (addPC.south east);
%\draw[-, black!80, thick] (addPC.north east) -- (addPC.south west);
% add Excitatory
\node[roundNode] (addExc) [left=0.8cm of PSPEIN.west]{};
\draw[-, black!80, thick] (addExc.west) -- (addExc.east);
\draw[-, black!80, thick] (addExc.north) -- (addExc.south);

% add PC -> Sigm PC -> PSP PC
\draw[-{Stealth[scale=1.5]}] (addPC.east) -- (SigmPC.west)node[coordinate, pos=0.5](measurepoint){};
\draw[-{Stealth[scale=1.5]}] (SigmPC.east) -| (c1.south);
\draw[-{Stealth[scale=1.5]}] (SigmPC.east) -| (c3.north);


% y0 -> C1 -> Sigm EIN
\draw[-] (c1.north) |- (in_PSP_PC_e.center);
\draw[-{Stealth[scale=0.5]}] (in_PSP_PC_e.center) |- (PSPPC1.east);
\draw[-{Stealth[scale=0.5]}] (in_PSP_PC_e.center) |- (PSPPC2.east);
\draw[-{Stealth[scale=0.5]}] (PSPPC1.west) -- (PSPPCw1.east);
\draw[-{Stealth[scale=0.5]}] (PSPPC2.west) -- (PSPPCw2.east);
\draw[-{Stealth[scale=0.5]}] (PSPPCw1.west) -| (out_PSP_PC_e.north);
\draw[-{Stealth[scale=0.5]}] (PSPPCw2.west) -| (out_PSP_PC_e.south);
\draw[-{Stealth[scale=1.5]}] (out_PSP_PC_e.west) -- (SigmEIN.east) node[pos=0.4, above=0.1cm, fill=pyratesgreen!15,
draw=pyratesgreen, text=pyratesgreen]{\tiny$PSP_{EIN}$};

% y0 -> C3 -> Sigm IIN
\draw[-] (c3.south) |- (in_PSP_PC_i.center);
\draw[-{Stealth[scale=0.5]}] (in_PSP_PC_i.center) |- (PSPPCI1.east);
\draw[-{Stealth[scale=0.5]}] (in_PSP_PC_i.center) |- (PSPPCI2.east);
\draw[-{Stealth[scale=0.5]}] (PSPPCI1.west) -- (PSPPCIw1.east);
\draw[-{Stealth[scale=0.5]}] (PSPPCI2.west) -- (PSPPCIw2.east);
\draw[-{Stealth[scale=0.5]}] (PSPPCIw1.west) -| (out_PSP_PC_i.north);
\draw[-{Stealth[scale=0.5]}] (PSPPCIw2.west) -| (out_PSP_PC_i.south);
\draw[-{Stealth[scale=1.5]}] (out_PSP_PC_i.west) -- (SigmIIN.east) node[pos=0.4, below=0.1cm, draw=pyratesdarkred,
fill=pyratesdarkred!15, text=pyratesdarkred]{\tiny$PSP_{IIN}$};


% Sigm EIN -> c2 -> add EXC
\draw[-{Stealth[scale=1.5]}] (SigmEIN.west) -- (c2.east);
\draw[-{Stealth[scale=1.5]}] (c2.west) -| (addExc.north) node[pos=0.9, right]{\small$+$};
% external -> add EXC
\draw[-{Stealth[scale=1.5]}] (ext.east) -- (addExc.west) node[pos=0.9, above]{\small$+$};

% add EXC -> PSP EIN
\draw[-] (addExc.east) -- (in_PSP_EIN.center);
% inside PSP EIN
\draw[-{Stealth[scale=0.5]}] (in_PSP_EIN.center) |- (PSP_EIN1.west);
\draw[-{Stealth[scale=0.5]}] (in_PSP_EIN.center) |- (PSP_EIN2.west);
\draw[-{Stealth[scale=0.5]}] (PSP_EIN1.east) -- (PSP_EINw1.west);
\draw[-{Stealth[scale=0.5]}] (PSP_EIN2.east) -- (PSP_EINw2.west);
\draw[-{Stealth[scale=0.5]}] (PSP_EINw1.east) -| (out_PSP_EIN.north);
\draw[-{Stealth[scale=0.5]}] (PSP_EINw2.east) -| (out_PSP_EIN.south);
% PSP EIN -> add PC
\draw[-{Stealth[scale=1.5]}, fill=none] (out_PSP_EIN.east) -| (addPC.north) node[pos=1, left]{\small$+$} node[pos=0.4, above=0.1cm, draw=pyratespurple, fill=pyratespurple!15, text=pyratespurple]{\tiny$PSP_{PC_E}$};

% Sigm IIN -> C4 -> PSP IIN
\draw[-{Stealth[scale=1.5]}] (SigmIIN.west) -- (c4.east);
\draw[-] (c4.west) -| (inpIPSP.center);
\draw[-] (inpIPSP.center) -- (in_PSP_IIN.center);
% inside PSP IIN
\draw[-{Stealth[scale=0.5]}] (in_PSP_IIN.center) |- (PSP_IIN1.west);
\draw[-{Stealth[scale=0.5]}] (in_PSP_IIN.center) |- (PSP_IIN2.west);
\draw[-{Stealth[scale=0.5]}] (PSP_IIN1.east) -- (PSP_IINw1.west);
\draw[-{Stealth[scale=0.5]}] (PSP_IIN2.east) -- (PSP_IINw2.west);
\draw[-{Stealth[scale=0.5]}] (PSP_IINw1.east) -| (out_PSP_IIN.north);
\draw[-{Stealth[scale=0.5]}] (PSP_IINw2.east) -| (out_PSP_IIN.south);
% PSP IIN -> add PC
\draw[-{Stealth[scale=1.5]}, fill=none] (out_PSP_IIN.east) -| (addPC.south) node[pos=1, left]{\small$-$} node[pos=0.4, below=0.1cm, draw=pyratesorange, fill=pyratesorange!10, text=pyratesorange]{\tiny$PSP_{PC_I}$};

% electrode
\draw (measurepoint.north) -- (-0.9,1)node[coordinate,pos=0.9](a){} -- (-0.6,0.9)node[coordinate, pos=0.5](b){} -- cycle;
\node (signal)[above right=0.05cm and 2.0cm of a]{\tiny$PSP_{PC}$};
\draw (b.center) |- (signal.west);


%\begin{scope}[shift={(PSPPC1.south west)}]
%      \begin{axis}[yscale=0.03, xscale=0.1,
%            axis x line=none,
%            axis y line=none,
%            domain=0:140,
%            samples=1001,
%            xticklabels=\empty,
%          ]
%          \addplot [green!80] {0.325*x*e^(-0.1*x)};
%        \end{axis}
%\end{scope}
%
%\begin{scope}[shift={(PSPPCI.south west)}]
%      \begin{axis}[yscale=0.03, xscale=0.16,
%            axis x line=none,
%            axis y line=none,
%            domain=0:1400,
%            samples=1001,
%            xticklabels=\empty,
%          ]
%          \addplot [green!80] {0.325*x*e^(-0.1*x)};
%        \end{axis}
%\end{scope}
%
%\begin{scope}[shift={(PSPEIN.south west)}]
%      \begin{axis}[yscale=0.03, xscale=0.16,
%            axis x line=none,
%            axis y line=none,
%            domain=0:140,
%            samples=1001,
%            xticklabels=\empty,
%          ]
%          \addplot [green!80] {0.325*x*e^(-0.1*x)};
%        \end{axis}
%\end{scope}
%
%\begin{scope}[shift={(PSPIIN.south west)}]
%      \begin{axis}[yscale=0.12, xscale=0.16,
%            axis x line=none,
%            axis y line=none,
%            domain=0:140,
%            samples=1001,
%            xticklabels=\empty,
%          ]
%          \addplot [red!80] {1.1*x*e^(-0.05*x)};
%        \end{axis}
%\end{scope}

\begin{pgfonlayer}{bg}

    \node[draw=cyan!40, fill=cyan!2, inner xsep=0.15cm, inner ysep=0.15cm,
      fit=(PSPEIN) (SigmPC) (PSPIIN) ] {};
%    \filldraw [fill=cyan!2,draw=cyan!40]
%        ($ (PSPEIN.center) + (-0.4,0.8) $)
%        rectangle ($ (PSPIIN.center) + (3.8,-0.8) $);
    \node [below=0.2cm of SigmPC, text=cyan]{PC};

    \node[draw=green!40, fill=green!2, inner xsep=0.15cm, inner ysep=0.15cm,
      fit=(PSPPC) (SigmEIN) ] {};
%    \filldraw [fill=green!2,draw=green!40]
%        ($ (SigmEIN.north) + (-0.7,0.4) $)
%        rectangle ($ (PSPPC.south) + (0.4,-0.2) $);
    \node [above=1.4cm of SigmPC, text=green]{EIN};
    
    \node[draw=red!40, fill=red!2, inner xsep=0.15cm, inner ysep=0.15cm,
      fit=(PSPPCI) (SigmIIN) ] {};
%    \filldraw [fill=red!2,draw=red!40]
%        ($ (SigmIIN.north) +  (-0.7,0.2) $)
%        rectangle ($ (PSPPCI.south) + (0.4,-0.4) $);
    \node [below=1.4cm of SigmPC, text=red]{IIN};
\end{pgfonlayer}

\end{tikzpicture}
%\caption{\textbf{David \& Friston Block Diagram as implemented in PyRates}}
%\label{fig:pyratesDFBlock}
%\end{figure}
%
%For the final analysis, we use the David \& Friston Model with 2 subpopulations for each PSP-Block.
%The parameters for the subpopulations are defined as follows:
%\begin{gather*}
%    \text{slow subpopulations:}\quad \tau_{e_1}=10.8ms, \tau_{i_1}=22ms\\
%    \text{fast subpopulations:}\quad \tau_{e_2}=4.6ms, \tau_{i_2}=2.9ms\\
%\end{gather*}
%

\todo{CLARIFY ALL THE HEADINGS!!!!! APPLIES TO WHOLE THESIS!!!}
\section{Exploring realistic parameter ranges}\label{sec:simulation-over-the-parameter-space}

When simulating a linear increase and subsequent decrease over the selected parameter space for
$ \lambda \in \left[ 1, 3 \right] $ (see Fig~\ref{fig:sedation_sim_jr} \textbf{A}),
the following phenomena can be observed:

\newtoggle{drawLocRoc}

\toggletrue{drawLocRoc}
\def\simRunName{JR_LONGER_INIT}
\def\locStart{1.1}
\def\locST{6.61}
\def\locEnd{2.07}
\def\locET{18.89}
\def\rocStart{2.0}
\def\rocST{48.63}
\def\rocEnd{1.00}
\def\rocET{62.0}
\subsection{Phenomenology of the Basic JR Model}
    In the baseline state ($\lambda = 1$), the model produces the expected well-defined alpha-activity,
    with a sharp frequency peak at $\SI{12}{\hertz}$.
\begin{remark}[Spectrogram Amplitude]
    At this point it should be noted,
    that even though the spectrogram (Fig~\ref{fig:sedation_sim_jr} \textbf{B}) appears to contain a frequency range
    from roughly $6-18\SI{}{\hertz}$,
    this is misleading.
    The extremely high-amplitude $\SI{12}{\hertz}$ peak has noisy flanks with low amplitudes
    (compare to the spectrum in Fig.~\ref{fig:initial_oscilations} \textbf{C} \& \textbf{D}),
    which are highlighted by the chosen colormap and value range.
    Their relative contribution to the signal is close to insignificant in this case.
\end{remark}
    Increasing the simulated propofol levels first leads to the single dominant frequency at $\SI{12}{\hertz} $
    descending towards $\SI{10}{\hertz} $.
    When reaching $\lambda \approx \locStart $,
    there is an onset of heavy oscillations, characterized by dramatically increasing signal amplitude
    with frequencies peaking at around $0-10, 12-20$ and $ \SI{30}{\hertz} $,
    but extending up to $\sim \SI{35}{\hertz} $.
    Further increase of the concentration shifts the frequency peaks of the disturbed signal slowly towards lower
    values, also decreasing their distance to each other.
    Additionally, higher values of $\lambda$ appear to linearly decrease the mean signal voltage.
    Another sudden change occurs around $\lambda \approx \locEnd $,
    where the disturbances disappear again,
    with the system having apparently reached a different stable state with very low activity.
    Signal amplitude is drastically reduced, with remaining activity mostly below $\SI{10}{\hertz}$.
    The single dominant frequency has disappeared completely.
    From there on, the mean signal voltage slowly continues to slightly decrease as before,
    however the frequency distribution appears to have settled.
    After reaching peak dosage at $\lambda = 3.0$, maintaining it for a few seconds has no further effects.
    Subsequent concentration decrease has the expected reverse effects:
    at first only the signal voltage slightly increases,
    then disturbances begin to form again at $\lambda \approx \rocStart$.
    On the return path, effects roughly mirror the induction.
    It takes a few seconds after $\lambda$ returns to its initial value until the initial state is restored.

\begin{figure}[H]
\begin{tikzpicture}
\pgfplotsset{
        %% Axis
            every x tick label/.append style={font=\tiny, yshift=0.5ex},
            every y tick label/.append style={font=\tiny, xshift=0.5ex},
            scale only axis,
            width=0.9\linewidth,
            height=3cm,
            every axis/.append style={
                line width=1pt,
                tick style={line width=0.8pt},
            },
            %% X-Axis
            xmin=1.0, xmax=66.0,
        }
        \begin{groupplot} [
                group style={
                    group size=1 by 3,
                    vertical sep=2mm,
                    xlabels at=edge bottom,
                    xticklabels at=edge bottom,
                },
                yticklabel style={
                    /pgf/number format/fixed,
                    /pgf/number format/precision=2
                },
                legend style={nodes={scale=0.8, transform shape}, thin},
                legend image post style={scale=0},
                xlabel=$t(s)$
            ]
            \nextgroupplot[ylabel style={align=center}, ylabel=\begin{tiny}decay-time factor\end{tiny}\\ $\lambda$,
                           ymin=0.6, ymax=3.1, grid style={dashed,black!20}, grid=major]
            \addplot [name path=factors,line width=.5pt,solid, cyan]
            table[x=x, y=y ,col sep=comma, each nth point=100]{data/full_sedation_sim/\simRunName _factors.csv};

            \legend{\textbf{(a) Propofol concentration}};
        \iftoggle{drawLocRoc}{
                \draw [red, dotted] ([xshift=0.0cm]axis cs:0,\locStart) -- ([yshift=0.0cm]axis cs:\locST,\locStart) ;
                \draw [red, dotted] ([xshift=0.0cm]axis cs:0,\locEnd) -- ([yshift=0.0cm]axis cs:\locET,\locEnd)
            node[near
            end, above,font=\tiny]{LOC $\approx [\locStart,\locEnd]$};
                \draw [red, dotted] ([xshift=0.0cm]axis cs:\locET,0) -- ([yshift=0.0cm]axis cs:\locET,\locEnd);
                \draw [red, dotted] ([xshift=0.0cm]axis cs:\locST,0) -- ([yshift=0.0cm]axis cs:\locST,\locStart);

                \draw [green, dotted] ([xshift=0.0cm]axis cs:66.0,\rocEnd) -- ([yshift=0.0cm]axis cs:\rocET,\rocEnd) ;
                \draw [green, dotted] ([xshift=0.0cm]axis cs:66.0,\rocStart) -- ([yshift=0.0cm]axis cs:\rocST,\rocStart)
            node[near
                       end,above,font=\tiny]{ROC $\approx [\rocStart,\rocEnd]$};
                \draw [green, dotted] ([xshift=0.0cm]axis cs:\rocST,0) -- ([yshift=0.0cm]axis cs:\rocST,\rocStart);
                \draw [green, dotted,line width=.5pt] ([xshift=0.0cm]axis cs:\rocET,0) -- ([yshift=0.0cm]axis cs:\rocET,\rocEnd);
        }

            \nextgroupplot[ylabel=$mV$]
            \addplot [line width=.5pt,solid, cyan]
            table[x=x, y=y ,col sep=comma, each nth point=2]
            {data/full_sedation_sim/\simRunName .csv};

            \legend{\textbf{(b) Signal}};

            \nextgroupplot[ymin=0, ymax=40,
                            ylabel=$Hz$,
                            height=5cm
            ]
              \addplot graphics [includegraphics cmd=\pgfimage,
        xmin=1.0,%xmin=0.899,
        xmax=66.0,% xmax=38.899,
        ymin=0, ymax=40]
            {data/full_sedation_sim/\simRunName -img0.png};
            \legend{\textbf{(c) Spectrogram}};

        \end{groupplot}
     \begin{groupplot} [group style={group size=1 by 3,vertical sep=2mm}]
            \nextgroupplot[axis y line=right, axis line style={-}, ymin=0.6, ymax=3.1, ylabel=$\sim c_e
            (\SI{}{\micro\molar})$,
            yticklabels ={0,5,10,20,30}, ytick={1,1.81,2.164,2.558,2.8},
            axis x line=none]
            \nextgroupplot[axis y line=none, axis x line=none]
            \nextgroupplot[axis y line=right, axis line style={-}, xshift=0.5cm, ymin=0, ymax=40,
                ylabel=$dB$,
                yticklabels ={-110,-85,-40}, ytick={0, 20, 40},
                height=5cm,axis x line=none]
        \end{groupplot}

              \node [above right] at (12.8cm, -8.53cm) {
            \includegraphics[width=0.35cm, height=5cm]
            {data/full_sedation_sim/\simRunName -img1.png}};
\end{tikzpicture}
\caption{\textbf{Simulation of a sedation in the JR Model:} \\
        \textbf{A):} timeline of the simulated IPSP stretch factor $\lambda$ (roughly representing $c_e$) \\
        \textbf{B):} timeline of the simulated signal \\
        \textbf{C):} spectrogram
}\label{fig:sedation_sim_jr}
\end{figure}

\toggletrue{drawLocRoc}
\def\simRunName{DF_LONGER_INIT}
\def\locStart{1.85}
\def\locST{16.17}
\def\locEnd{2.05}
\def\locET{18.72}
\def\rocStart{1.95}
\def\rocST{49.3}
\def\rocEnd{1.5}
\def\rocET{55.03}
\subsection{Phenomenology of the David Friston Extension}
    The subpopulation extension of the David \& Friston model introduces the mixture of slow and fast kinetics within
    a population.
    This results in an initial frequency spectrum,
    that lacks the distinct high-amplitude peak and instead has a wider distribution with far lower amplitude.
\begin{remark}[Spectrogram Amplitude]
    Fig.~\ref{fig:sedation_sim_jr} \textbf{C} and Fig.~\ref{fig:sedation_sim_df} \textbf{C} have the same
    colormap and amplitude-range.
    In contrast to the spectrogram of the JR-Model,
    the low-amplitude activity is significant from the start here,
    as it makes up the whole baseline signal.
\end{remark}

    The DF model initially produces a dominant frequency range at $10-20 \SI{}{\hertz} $,
    which slowly shifts towards $ 5-10 \SI{}{\hertz} $ (\textbf{C}) when $\lambda$ starts to increase.
    The mean signal voltage concurrently decreases,
    while oscillation-amplitude is maintained (Fig~\ref{fig:sedation_sim_df} \textbf{B}).
    The system appears to be in a stable state until $ \lambda \approx \locStart $,
    where the sudden onset of dramatically
    increasing signal amplitude with strongly pronounced activity below $ \SI{25}{\hertz} $
    creates multiple strong frequency peaks.
    Unlike in the JR model, the oscillations do not exceed  $ \SI{30}{\hertz} $.
    Further increasing $\lambda$ has the same minimal effects on the disturbed signal,
    as the increase had before exiting the stable state.
    The heavy oscillations remain until $\lambda \approx \locEnd $,
    where the dominant frequencies move below $\SI{10}{\hertz}$.
    Continuing, the signal voltage slowly decreases as before,
    however the frequency distribution appears to have settled.
    Maintaining peak dosage has no further effects.
    Decreasing the simulated propofol levels again first increases the mean signal voltage,
    until the stable state dissolves into heavy oscillations around $\lambda \approx \rocStart$.
    The unstable state prevails until $\lambda$ reaches $\approx \rocEnd$.


\begin{figure}[H]
\begin{tikzpicture}
\pgfplotsset{
        %% Axis
            every x tick label/.append style={font=\tiny, yshift=0.5ex},
            every y tick label/.append style={font=\tiny, xshift=0.5ex},
            scale only axis,
            width=0.9\linewidth,
            height=3cm,
            every axis/.append style={
                line width=1pt,
                tick style={line width=0.8pt},
            },
            %% X-Axis
            xmin=1.0, xmax=66.0,
        }
        \begin{groupplot} [
                group style={
                    group size=1 by 3,
                    vertical sep=2mm,
                    xlabels at=edge bottom,
                    xticklabels at=edge bottom,
                },
                yticklabel style={
                    /pgf/number format/fixed,
                    /pgf/number format/precision=2
                },
                legend style={nodes={scale=0.8, transform shape}, thin},
                legend image post style={scale=0},
                xlabel=$t(s)$
            ]
            \nextgroupplot[ylabel style={align=center}, ylabel=\begin{tiny}decay-time factor\end{tiny}\\ $\lambda$,
                           ymin=0.6, ymax=3.1, grid style={dashed,black!20}, grid=major]
            \addplot [name path=factors,line width=.5pt,solid, cyan]
            table[x=x, y=y ,col sep=comma, each nth point=100]{data/full_sedation_sim/\simRunName _factors.csv};

            \legend{\textbf{(a) Propofol concentration}};
        \iftoggle{drawLocRoc}{
                \draw [red, dotted] ([xshift=0.0cm]axis cs:0,\locStart) -- ([yshift=0.0cm]axis cs:\locST,\locStart) ;
                \draw [red, dotted] ([xshift=0.0cm]axis cs:0,\locEnd) -- ([yshift=0.0cm]axis cs:\locET,\locEnd)
            node[near
            end, above,font=\tiny]{LOC $\approx [\locStart,\locEnd]$};
                \draw [red, dotted] ([xshift=0.0cm]axis cs:\locET,0) -- ([yshift=0.0cm]axis cs:\locET,\locEnd);
                \draw [red, dotted] ([xshift=0.0cm]axis cs:\locST,0) -- ([yshift=0.0cm]axis cs:\locST,\locStart);

                \draw [green, dotted] ([xshift=0.0cm]axis cs:66.0,\rocEnd) -- ([yshift=0.0cm]axis cs:\rocET,\rocEnd) ;
                \draw [green, dotted] ([xshift=0.0cm]axis cs:66.0,\rocStart) -- ([yshift=0.0cm]axis cs:\rocST,\rocStart)
            node[near
                       end,above,font=\tiny]{ROC $\approx [\rocStart,\rocEnd]$};
                \draw [green, dotted] ([xshift=0.0cm]axis cs:\rocST,0) -- ([yshift=0.0cm]axis cs:\rocST,\rocStart);
                \draw [green, dotted,line width=.5pt] ([xshift=0.0cm]axis cs:\rocET,0) -- ([yshift=0.0cm]axis cs:\rocET,\rocEnd);
        }

            \nextgroupplot[ylabel=$mV$]
            \addplot [line width=.5pt,solid, cyan]
            table[x=x, y=y ,col sep=comma, each nth point=2]
            {data/full_sedation_sim/\simRunName .csv};

            \legend{\textbf{(b) Signal}};

            \nextgroupplot[ymin=0, ymax=40,
                            ylabel=$Hz$,
                            height=5cm
            ]
              \addplot graphics [includegraphics cmd=\pgfimage,
        xmin=1.0,%xmin=0.899,
        xmax=66.0,% xmax=38.899,
        ymin=0, ymax=40]
            {data/full_sedation_sim/\simRunName -img0.png};
            \legend{\textbf{(c) Spectrogram}};

        \end{groupplot}
     \begin{groupplot} [group style={group size=1 by 3,vertical sep=2mm}]
            \nextgroupplot[axis y line=right, axis line style={-}, ymin=0.6, ymax=3.1, ylabel=$\sim c_e
            (\SI{}{\micro\molar})$,
            yticklabels ={0,5,10,20,30}, ytick={1,1.81,2.164,2.558,2.8},
            axis x line=none]
            \nextgroupplot[axis y line=none, axis x line=none]
            \nextgroupplot[axis y line=right, axis line style={-}, xshift=0.5cm, ymin=0, ymax=40,
                ylabel=$dB$,
                yticklabels ={-110,-85,-40}, ytick={0, 20, 40},
                height=5cm,axis x line=none]
        \end{groupplot}

              \node [above right] at (12.8cm, -8.53cm) {
            \includegraphics[width=0.35cm, height=5cm]
            {data/full_sedation_sim/\simRunName -img1.png}};
\end{tikzpicture}
\caption{\textbf{Simulation of a sedation in the DF Model:} \\
        \textbf{A):} timeline of the simulated IPSP stretch factor $\lambda$ (roughly representing $c_e$) \\
        \textbf{B):} timeline of the simulated signal \\
        \textbf{C):} spectrogram
}\label{fig:sedation_sim_df}
\end{figure}

\section{Similarities and Differences}
\todo{highlight relevant ("striking") differences and similarities}
    Both models share some key features during these simulations:
    Three, distinguishable states can be observed. 
    The initial stable state is followed by a state that is defined by its strong harmonics(?) and dramatically increased
    signal amplitude.
    From there, the system transitions into a low-power state with low frequencies,
    which seems to be mostly saturated and changes little to further increasing agent concentration.
    Induction and Emergence are asymmetrical, even if only marginally in the JR model.
    The system predicts that the frequency range below $\SI{25}{\hertz}$ receives a temporary amplitude boost during
    the phase transitions, which disappears while the parameter changes continue in the same direction.
    In the `unconscious` state, frequency distribution stabilizes independent of further increasing decay-time
    (\textbf{slow-wave-activity saturation}?~\cite{ni_mhuircheartaigh_slow_wave_2013}).

There are also a few striking differences between the two simulation sessions:
By design,
the JR model produces exclusively near-sinusoid single-peak alpha-activity in the initial state ($\lambda = 1$),
while the DF model creates a range of alpha- and beta-activity.
The JR model spends significant amounts of the session in the second, high-amplitude state,
while these sections are significantly shorter for the DF model.




\todo{transition into discussion}

