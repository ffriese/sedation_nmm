\chapter{Results}\label{ch:results}


\section{General features of the simulated data}\label{sec:general-features-of-the-simulated-data}

The NMM is configured to simulate data with a step-size of $1 ms$, yielding a $1000 Hz$ signal.
In its initial state, the system reacts with high-amplitude oscillations to the `disturbance` of the random input.
However, the signal usually stabilizes quickly and exhibits the expected behaviour.
Thus, the very first second of simulated data is always discarded before further analysis.
When generating data continuously (without re-initialization of the state-variables),
this problem only occurs once in the very beginning.
System parameters can be changed after each simulated section and the signal reacts accordingly.
Changing parameters abruptly can also result in disturbances and destabilize the signal.

\begin{figure}[H]
    \centering
    \begin{tikzpicture}
        \pgfplotsset{
        %% Axis
            scale only axis,
            width=0.4\linewidth,
            height=4cm,
            every axis/.append style={
                line width=1pt,
                tick style={line width=0.8pt},
                %   grid style={dashed, black!20},
                %  grid=major,
            },
%               %% X-Axis
            xmin=-0.1,
            xmax=3,
        }
        \begin{groupplot}
            [
            group style={
                group size=2 by 1,
                vertical sep=2mm,
                horizontal sep=15mm,
                xlabels at=edge bottom,
                xticklabels at=edge bottom,
            },
            yticklabel style={
                /pgf/number format/fixed,
                /pgf/number format/precision=2
            },
            legend style={nodes={scale=0.8, transform shape}, thin},
            legend image post style={scale=0},
            xlabel=$s$,
            ]
            \nextgroupplot[ylabel=$mV$]
            \addplot [line width=1pt,solid, cyan] table[x=x,y=y ,col sep=comma]{data/methodology/uncut.csv};
            \legend{\textbf{A} uncut data};
            \nextgroupplot[xmin=1.0]
            \addplot [line width=1pt,solid, cyan] table[x=x,y=y ,col sep=comma]{data/methodology/uncut.csv};
            \legend{\textbf{B} cut data};

        \end{groupplot}
    \end{tikzpicture}

    \caption{\textbf{Processing of simulated data.}\\ \textbf{A \& B:} removing initial oscillations.}
    \label{fig:initial_oscilations}
\end{figure}


\section{Visualization}\label{sec:visualization}




\begin{figure}[H]
\includegraphics[width=15cm]{Figures/temp_sim_results}
\caption{\textbf{Power Spectral Density of Simulation Results:} Reduction of the connectivity parameter $C$
    from $135$ to $108$ shows a tendency of reducing the strength of the frequency-bands above 12--14Hz,
    while increasing it below that value - especially around 8--12Hz.}
\label{fig:sim_results1}
\end{figure}

\todo{objective description of simulation results}